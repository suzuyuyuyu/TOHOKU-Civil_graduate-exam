\documentclass[10pt, a4paper]{ltjsarticle}
% -------------------------------------------------------%

% パッケージのインポート

\usepackage{graphicx} % 図の挿入(includegraphics)
\usepackage{wrapfig}
\usepackage{booktabs} % tableのmidrule
\usepackage{multirow} % tableのmultirow
\usepackage{float} % [H]で厳密に位置を固定

\usepackage{titlesec} % タイトルの書式を変える(titleformat)

\usepackage{bm}
\usepackage{amsmath} % 数式用
\usepackage{amssymb} % もっと数式記号(iintとか)
\usepackage{mathtools} % もっと数式(アクセント)
\usepackage{accents} % \undertildeで下付きチルダ
\usepackage{nccmath} % 数式左寄せ環境fleqn
\usepackage{empheq} % わからん

\usepackage{amsthm} % 定理,証明など

\usepackage{siunitx} % 単位書き方(si)

\usepackage[version=4]{mhchem} % 化学式
\usepackage{chemfig}
\usepackage{url}

\usepackage{enumitem}
\usepackage{comment} % コメントアウト(comment環境)

\usepackage{framed} % 左線
\usepackage{pict2e} % ベンゼン
\usepackage{ascmac} % itembox
\usepackage{fancybox} % itembox

% ページスタイルのため
\usepackage{fancyhdr}
\usepackage{lastpage}
\usepackage{color}

% -------------------------------------------------------%

% ページ番号に西暦を入れる用のカウンター
\newcounter{yearcounter}

% yearcounterとpageを変更
% セクション始まりにこの2つを書けばページ番号を振り直せる
% \setcounter{yearcounter}{2020}
% \setcounter{page}{1}

% これは多分使わない
% yearcounterを1増加させる
\newcommand{\incyearcounter}{\stepcounter{yearcounter}}

% -------------------------------------------------------%

% 作ったけど使ってません

% \pagestyle{plain} % empty:ページ番号削除
% ページスタイルはドキュメント内で指定できる

% メインのページスタイル 2020-1
\fancypagestyle{mainPS}{%
\lhead{\leftmark}%ヘッダ左を空に
\rhead{\rightmark}%ヘッダ右を空に
\cfoot{\thepage}%フッタ中央に"今のページ数/総ページ数"を設定
\renewcommand{\headrulewidth}{0.0pt}%ヘッダの線を消す
}

% appendixのページスタイル appendix-1
\fancypagestyle{appendixPS}{%
\lhead{}
\rhead{}
\cfoot{appendix\:\raisebox{1pt}{-}\thepage}
}


% titlepageのページスタイル
\fancypagestyle{titlePS}{%
\lhead{empty}
\rhead{empty}
\cfoot{empty}
}

% -------------------------------------------------------%

% セクション・サブセクションの見出しのサイズ
% \titleformat*{\section}{\Large\bfseries} % サイズ・太字
% \titleformat*{\subsection}{\large\bfseries}
\titleformat{\section}[block]
{
  % sectionとしたら絶対につく装飾
  % \noindent
  {\colorbox{black}{\begin{picture}(0,10)\end{picture}}}
  \hspace*{-13pt}
  \raisebox{-4.5pt}{\begin{picture}(0,0)
    \put(0,0){\color{black}\line(1,0){300}}
  \end{picture}}
  \hspace*{4pt}
}
{
  % section*では消える場所
  % \hspace{0pt}
  % {\normalfont \Large\bfseries}
}{0pt}
{
  \normalfont \Large\bfseries 
}
[ 
]

\titleformat{\subsection}[block]
{%
  \raisebox{-5pt}%
  {\linethickness{1.5pt}%
  {\begin{picture}(5,20)%
    \put(0,0){\line(0,1){20}}%
  \end{picture}}}%
  \raisebox{-4pt}
  {\linethickness{0.5pt}%
  {\begin{picture}(0,20)%
    \put(0,0){\line(0,1){18}}%
  \end{picture}}}
}{%
}{0pt}
{
  \hspace*{0pt}\normalfont \large\bfseries 
}
[ 
]


% その他コマンド
\renewcommand{\figurename}{図}
\renewcommand{\thefigure}{\arabic{figure}:}
\renewcommand{\tablename}{表}
\renewcommand{\thetable}{\arabic{table}:}

% 再帰的にコマンドを改変する
% \let\origincaption\caption
% \renewcommand{\caption}[1]{\origincaption{#1}}

% -------------------------------------------------------%

\makeatletter

\newcommand{\reference}[0]{\setlength{\hangindent}{18pt}\noindent}

\newcommand{\yearn}[0]{\number\year}
\newcommand{\monthn}[0]{\number\month}
\newcommand{\dayn}[0]{\number\day}

% 数式や図表のref
\renewcommand{\refeq}[1]{\eqref{#1}式}
\newcommand{\reffig}[1]{図\ref{#1}}
\newcommand{\reftbl}[1]{表\ref{#1}}

% よく使う単位
\newcommand{\degC}[0]{\mathrm{{}^\circ \hspace*{-0.5pt} C}}
\renewcommand{\deg}[0]{\mathrm{{}^\circ}}

% よく使う演算子
\renewcommand{\Vector}[1]{{\mbox{\boldmath$#1$}}}
\newcommand{\tensor}[1]{\undertilde{#1}}
\renewcommand{\rm}[1]{\mathrm{#1}}

% displaystyle math mode
\newcommand{\dm}[1]{
  $\displaystyle #1 $
}
\newcommand{\disp}{\displaystyle}
  
% よく使う記号
\newcommand{\bA}{\bm{A}}
\newcommand{\bB}{\bm{B}}
\newcommand{\bE}{\bm{E}}
\newcommand{\bC}{\bm{C}}
\newcommand{\bH}{\bm{H}}
\newcommand{\bL}{\bm{L}}
\newcommand{\bU}{\bm{U}}
\newcommand{\bP}{\bm{P}}

\newcommand{\ba}{{\bm{a}}}
\newcommand{\bb}{{\bm{b}}}
\newcommand{\bc}{{\bm{c}}}
\newcommand{\bd}{{\bm{d}}}
\newcommand{\be}{{\bm{e}}}

\newcommand{\bn}{{\bm{n}}}

\newcommand{\bx}{{\bm{x}}}
\newcommand{\by}{{\bm{y}}}

\newcommand{\bu}{{\bm{u}}}
\newcommand{\bv}{{\bm{v}}}
% \newcommand{\zerov}{\bm{0}}

% -------------------------------------------------------%

% 数式番号にセクション番号を併記する
\renewcommand{\theequation}{\thesection.\arabic{equation}}
\makeatletter
\@addtoreset{equation}{section} 
\makeatother

% -------------------------------------------------------%

% 問題文の左の線の定義
\renewenvironment{leftbar}{%
\def\FrameCommand{\hspace{10pt}\vrule width 2pt \hspace{10pt}}%
\MakeFramed {\advance\hsize-\width \FrameRestore}}%
{\endMakeFramed}
\renewcommand{\thesubsubsection}{\large\textbf{問\arabic{subsubsection}}}

% -------------------------------------------------------%

% 定理スタイルの定義
\newtheoremstyle{mystyle}
  {\topsep}   % スペース上
  {\topsep}   % スペース下
  {\normalfont}  % 本文のフォント
  {0pt}       % インデント
  {\bfseries} % タイトルのフォント
  {.}         % 句読点
  {.5em}      % タイトルと本文のスペース
  {}          % タイトルのスペース
% 定理環境の作成
\theoremstyle{mystyle}
\newtheorem*{ans*}{解答}
\newtheorem*{other*}{別解}
\newtheorem*{supple*}{Supplement}
\newtheorem*{append*}{Appendix}

% -------------------------------------------------------%

% 数学文字表現
\DeclareMathOperator{\rank}{rank}


% -------------------------------------------------------%

% 数学記号
\renewcommand{\parallel}{/\!/}

% -------------------------------------------------------%

% 演算子
\newcommand{\ppar}[2]{\frac{\partial #1}{\partial #2}}
\renewcommand{\d}{\partial}

% -------------------------------------------------------%

% 問題(problem)
\newcommand{\prob}[1]{\subsubsection{}\begin{leftbar}\noindent#1\end{leftbar}}

% alignを書くのが面倒なとき
\newcommand{\al}[1]{\begin{align}#1\end{align}}

% 集合の{(x,y)|x<0}みたいに書くときの縦線
\newcommand{\relmiddle}{\mathrel{}\middle| \mathrel{}}


% 2:問題番号を表示しない 3:する
\setcounter{tocdepth}{2}
% \setcounter{tocdepth}{3}



\title{東北大学 土木系 院試 基礎科目}
\author{鈴木\thanks{https://github.com/suzuyuyuyu}}
\date{}

\begin{document}

\pagestyle{empty}
% タイトル
\maketitle
% 目次
\tableofcontents


\pagestyle{appendixPS}

% =============================================================
% 微分積分
% - - - - - - - - - - - - -

% ロピタルの定理
\newpage
\section{ロピタル(L'H\^{o}pital)の定理}

微分可能な関数$f,\,g$と$a\in (-\infty,\,+\infty)$について
\dm{
  \lim_{x\to a}\frac{f(x)}{g(x)}
}
という極限を考えるとき、
\begin{gather}
  \lim_{x\to a} f(x) = \lim_{x\to a} g(x) = \pm \infty \:\:\rm{or}\:\:0
\end{gather}
であり、$x$が$a$の除外近傍($x=a$を除く$a$近傍)で$g'(x)\neq 0$であれば
\begin{gather}
  \lim_{x\to a}\frac{f(x)}{g(x)} = \lim_{x\to a}\frac{f'(x)}{g'(x)}
\end{gather}
が成り立つ。これを\lhopital の定理という。


% 微分方程式の解法
\newpage
\section{微分方程式の解法}
常微分方程式の解法についていくつか紹介する。
\begin{enumerate}[label=\arabic*.]
  \item 一階の場合
  \begin{enumerate}[label=\roman*.]
    \item 変数分離
    \item 同次形
    \item ベルヌーイ形の微分方程式
    \item 完全微分方程式と積分因子法
  \end{enumerate}
  \item 二階の場合
  \begin{enumerate}[label=\roman*.]
    \item 斉次形にして階数低減法や特性方程式(補助方程式)を用いて一般解を求める。
    \item 非斉次形の特解を未定係数法や定数変化法によって求めて
    足し合わせる。
  \end{enumerate}
\end{enumerate}

うち、完全微分方程式と積分因子法、未定係数法、定数変化法について述べる。


% 完全微分方程式と積分因子法
\newpage
\section{完全微分方程式と積分因子法}
\begin{gather}
  P(x,y)dx + Q(x,y)dy = 0 \\
  \ppar{P}{y} = \ppar{Q}{x} \quad(\text{可積分条件})
\end{gather}
と与えられる微分方程式を完全微分方程式という。

ある関数$U(=U(x,y))$に対する全微分は
\begin{gather}
  dU = \frac{\pd U}{\pd x}dx + \frac{\pd U}{\pd y}dy
\end{gather}
と与えられるので、微分方程式の$P,\,Q$について
\begin{gather}
  P(x,y) = \ppar{U}{x},\quad Q(x,y) = \ppar{U}{y}
\end{gather}
なる関数$U$が見つかれば、与えられた微分方程式は$dU = 0$すなわち
\begin{gather}
  U(x,y) = C \quad(C\text{は任意定数})
\end{gather}
と解を得る。

さらに、
\begin{gather}
  \ppar{P}{y} \neq \ppar{Q}{x}
\end{gather}
であっても、ある適当な関数$\grl(x,y)$が存在して、
\begin{gather}
  \ppar{(\grl P)}{y} = \ppar{(\grl Q)}{x} \label{eq:appendix-IF-intcond}
\end{gather}
が成り立つとき、あらたに関数$\hat{P}=\grl P,\,\hat{Q}=\grl Q$を用いて
\begin{gather}
  \hat{P}(x,y) dx + \hat{Q}(x,y) dy = 0 \\
  \ppar{\hat{P}}{y} = \ppar{\hat{Q}}{x}
\end{gather}
とできてこれは完全微分方程式である。
このときの関数$\grl(x,y)$を積分因子(Integrating Factor)という。
\refeq{eq:appendix-IF-intcond}について変形することで$\grl$の条件を考える。
\begin{gather}
  \ppar{\grl}{y}P + \ppar{P}{y}\grl = \ppar{\grl}{x}Q + \ppar{Q}{x}\grl
\end{gather}
ここで、$\grl=\grl(x)$であるような特殊な場合を仮定すれば
この$\grl(x,y)$についての偏微分方程式は$\grl(x)$についての常微分方程式
\begin{gather}
  \grl\Bigl(\ppar{P}{y} - \ppar{Q}{x}\Bigr) = \frac{d\grl}{dx}Q \\
  \Leftrightarrow
  \frac{1}{\grl}\frac{d\grl}{dx} = \frac{1}{Q(x,y)}\Bigl(\ppar{P}{y} - \ppar{Q}{x}\Bigr)
\end{gather}
となる。
左辺は$y$に依存しない($x$のみの式である)ので右辺\dm{\frac{1}{Q}(P_{y} - Q_{x})}
も$y$に依存しない式となれば積分因子$\grl$は$x$の関数$\grl(x)$としてよいということになる。

あるいは微分方程式を次のような一階線形常微分方程式に変形することを考える。
\begin{gather}
  \frac{dy}{dx} + P(x)y = Q(x) \label{eq:appendix-IF-linearODE}
\end{gather}
斉次形($Q(x) = 0$)のとき、これは変数分離形として解を得る。
非斉次形でかつ$Q$が$x$の関数のときは積分因子は一変数関数
$\grl(x)$として次のように求める。
両辺に$\grl(x)$をかけて
\begin{gather}
  \grl(x)\left\{ \frac{dy}{dx} + P(x)y \right\} = \grl(x)Q(x) \label{eq:appendix-IF-LODEgrl}
\end{gather}
としてこれが完全微分方程式となるための可積分条件
\begin{gather}
  \frac{1}{\grl}\frac{d\grl}{dx} = P(x) \label{eq:appendix-IF-linearODE-intcond}
\end{gather}
を解けば積分因子を得る。
また、\refeq{eq:appendix-IF-LODEgrl}について次式
\begin{gather}
  \grl(x)\left\{ \frac{dy}{dx} + P(x)y \right\} = \frac{d}{dx}(\grl(x)y) \label{eq:appendix-IF-LHScond}
\end{gather}
を満たすので、積分因子を\refeq{eq:appendix-IF-linearODE-intcond}によって計算して、
\begin{gather}
  \frac{d}{dx}(\grl(x)y) = \grl(x)Q(x)
\end{gather}
を解けば良い。一応簡単に証明しておく。

% []: この辺微分とか条件とか怪しい
\begin{proof}
  \refeq{eq:appendix-IF-LODEgrl}から\refeq{eq:appendix-IF-LHScond}を得ることを示す。
  \refeq{eq:appendix-IF-LODEgrl}は
  \begin{gather}
    \grl(x)\bigl(Q(x) - P(x)y\bigr)dx - \grl(x)dy = 0
  \end{gather}
  と変形できて、
  これが完全微分形であるとき可積分条件は
  \begin{gather}
    \ppar{}{y} \Bigl(\grl(x)\bigl(Q(x) - P(x)y\bigr)\Bigr) = - \ppar{\grl(x)}{x}
  \end{gather}
  であって、左辺は$y$に依存しないのでたしかに積分因子$\grl(x)$が存在して
  \begin{align}
    \frac{d\grl}{dx}
    &= \ppar{}{y}\bigl(\grl(x)P(x)y\bigr) - \ppar{(\grl Q)}{y} \\
    &= \grl(x)P(x) \label{eq:appendix-IF-grlODE}
  \end{align}
  すなわち
  \begin{gather}
    \frac{1}{\grl}\frac{d\grl}{dx} = P(x) \label{eq:appendix-IF-howtogrl}
  \end{gather}
  を満たす。(この変数分離形の微分方程式を解くことによって積分因子を計算できる。)

  \refeq{eq:appendix-IF-grlODE}の$\grl$に関する微分方程式の両辺を$y$倍して
  \dm{\grl\frac{dy}{dx}}を加えることで
  \begin{gather}
    \grl(x)\frac{dy}{dx} + \grl(x)P(x)y = y\frac{d\grl}{dx} + \grl(x)\frac{dy}{dx}
  \end{gather}
  を得る。この式は左辺を$\grl(x)$でくくり右辺を積の導関数と見れば、
  \refeq{eq:appendix-IF-LHScond}と同値である。
\end{proof}

\refeq{eq:appendix-IF-howtogrl}によって積分因子が得られれば、
\eqref{eq:appendix-IF-LODEgrl},\,\refeq{eq:appendix-IF-LHScond}より
\begin{gather}
  \frac{d}{dx}(\grl(x)y) = \grl(x)Q(x)
\end{gather}
として、この微分方程式を解けば得られる。

積分因子法は二階以上の線形微分方程式でも用いられる。

以上の議論は$x$と$y$について対称性があるので、
入れ替えた形で積分因子を$\grl(y)$のように仮定することもある。


% 未定係数法
\newpage
\section{未定係数法}
$y''+ay'+by=r(x)$のような微分方程式を
非斉次常微分方程式という。
通常、非斉次の場合は次のようにする。

\begin{table}[H]
\centering
\begin{tabular}{ccccc}
\toprule
  右辺$r(x)\propto$ & 特解$y_{\rm{p}}$ \\
\midrule
  $e^{ax}$ & $k e^{ax}$ \\
  $x^n$ & $k + k_1 x + \cdots + k_n x^n$ \\
  $\sin{ax},\cos{ax}$ & $k_1 \cos{ax} + k_2 \sin{ax}$ \\
\bottomrule
\end{tabular}\end{table}

ただし一般解にこれらが含まれている場合はその限りではない。
その場合は一般解が含まれなくなるまで$x$倍するような工夫が必要である。
たとえば、右辺が$e^x$であるような斉次方程式では
基本解として$e^x$と$e^{2x}$が得られたときには
特解を$xe^x$とおき、
基本解として$e^x$と$xe^{x}$が得られたときには
特解を$x^2e^x$とおく。

この方法を未定係数法といい、定数係数についての方程式を解けば特殊解が得られる。

未定係数法は適切に解の形を仮定しなくてはならないため、
右辺によっては良い仮定が見つからないことがある。
そこで定数変化法を用いることがある。


% 定数変化法
\newpage
\section{定数変化法}
前節の方法は,非斉次項 $r(x)$ が比較的単純で,特解の関数形を事前に予測できる場合にしか使え
ない。一方,定数変化法を使うことで,任意の関数 $r(x)$ に対して非斉次方程式 
$y'' + ay' + by = r(x)$ の特解を求めるための公式を作ることができる。
ただし,できる限り前節の方法を使った方が簡単に特解を求められるので注意すること。

\begin{itembox}[l]{非斉次方程式の特解の公式}
  \begin{gather}
    y(x) 
    = -\left( \int_{x_1}^{x}\frac{y_2(\hat{x})r(\hat{x})}{W(\hat{x})}\:d\hat{x} \right) y_1(x)
    + \left(\int_{x_2}^{x}\frac{y_1(\hat{x})r(\hat{x})}{W(\hat{x})}\:d\hat{x}\right)y_2(x) \label{23}
  \end{gather}
  ここで,
  
  \begin{itemize}
    \item $y_1(x),y_2(x)\colon$非斉次方程式の互いに独立な解 
    \item $x_1,x_2\colon$任意の定数
    \item $W(x)$は以下で定義される$x$の関数(ロンスキアン)
    \begin{gather}
      W(x) := y_1(x)y'_2(x) - y'_1(x)y_2(x) \label{24}
    \end{gather}
  \end{itemize}
\end{itembox}


公式 \eqref{23} を得るためには,非斉次方程式 の特解
 $y(x)$ をあえて次の形に書き表すところ
から計算をスタートする。

\begin{gather}
  y(x) = C_1(x)y_1(x) + C_2(x)y_2(x) \label{25}
\end{gather}

ただし,$C_1(x), C_2(x)$ は任意関数で,$y'' + ay' + by=0$ が満たされるように決める。

この式は単に,未知関数 $r(x)$ を別の未知関数 $C_1(x), C_2(x)$ で書き換えているに過ぎない。
さらに,特解 $y(x)$ を書き換えるだけなら未知関数が 1 つだけあれば十分である。
そこで,2 つの未知関数 $C_1(x), C_2(x)$ を結びつける関係式 \eqref{27} を後ほど導入し,
未知関数の個数を減らすことにする。

特解を求めるためには,上記の $y(x)$ を非斉次方程式 $y'' + ay' + by=0$ に代入して,式が満たされるように
$C_1(x), C_2(x)$ を決めればよい。そのために,まず $y'(x)$ を計算すると

\begin{gather}
  y'(x) 
  = (C_1(x)y_1(x) + C_2(x)y_2(x))'  \notag\\
  = C'_1(x)y_1(x) + C'_2(x)y_2 + C_1(x)y'_1(x) + C_2(x)y'_2(x) \label{26}
\end{gather}

以下の計算では未知関数 $C_1(x), C_2(x)$ を求めていくことになるが,
その際に $C_1, C_2$ の微分項がなる
べく少ない方が計算が簡単になる。
そこで,$C_1(x), C_2(x)$ が次の関係式を満たすと仮定する:

\begin{gather}
  C'_1(x)y_1(x) + C'_2(x)y_2(x) = 0 \label{27}
\end{gather}


こう仮定すると,$C_1(x)$ と $C_2(x)$ のどちらかを決めればもう一方が決まるため,
式 \eqref{25} の右辺に
含まれる未知関数の個数が実質的に一つになる。

仮定 \eqref{27} を課すと,$y'(x)$ の表式は
\begin{gather}
  y'(x) = C_1(x)y'_1(x) + C_2(x)y'_2(x) \label{28}
\end{gather}

このとき,$y''(x)$ は
\begin{gather}
  y''(x) 
  = (C_1(x)y'_1(x) + C_2(x)y'_2(x))' \notag\\
  = C'_1(x)y'_1(x) + C'_2(x)y'_2(x) + C_1(x)y''_1(x) + C_2(x)y''_2(x) \label{29}
\end{gather}

式 \eqref{28}, \eqref{29} より,式 \eqref{25} の $y(x)$ を
非斉次方程式 $y'' + ay' + by=0$ に代入したものは

\begin{gather}
  r(x) 
  = y'' + ay' + by \notag\\
  = C'_1y'_1 + C'_2y'_2 + C_1y''_1 + C_2y''2 
  + a(C_1y'_1 + C_2y'_2)+ C_1y_1 + C_2y_2 \notag\\
  = C_1(y''_1 + ay'_1 + by_1)+ C_2(y''2 + ay'_2 + by_2)+ C'_1y'_1 + C'_2y'_2 \notag\\
  = C'_1y'_1 + C'_2y'_2 \notag\\
  \therefore C'_1y'_1 + C'_2y'_2 = r(x)
  \quad(\text{ただし\:} C'_1y_1 + C'_2y_2 = 0) \label{30}
\end{gather}

あとは,式 \eqref{30} を解いて $C_1(x), C_2(x)$ を決めればよい。
まず,\eqref{30} の 2 つの式を組み合わせて $C_1(x)$ だけの式を作る。
$C'_1y_1 + C'_2y_2 = 0$ より,
$C'_2 = -(y_1/y_2)C'_1$ と書き換えられるので

\begin{gather}
  r(x) 
  = C'_1y'_1 + C'_2y'_2 \notag\\
  = C'_1 y'_1 - \frac{y_1}{y_2} C'_1 y'_2 \notag\\
  = \frac{y'_1y_2 - y_1y'_2}{y_2}C'_1 \notag\\
  = -\frac{W(x)}{y_2(x)}C'_1(x) \label{31}\\
  \therefore C'_1(x) = -\frac{y_2(x)r(x)}{W(x)} \notag\\
  \Rightarrow C_1(x) 
  = -\int_{x_1}^{x} \frac{y_2(\hat{x})r(\hat{x})}{W(\hat{x})}\:d\hat{x} \label{32}
\end{gather}

ただし,$x_1$ は任意の定数で,$C'_1(x)$ の式を積分する際に生じる積分定数に相当する。
また,式を単純化するために,式 \eqref{24} で定義されるロンスキアン $W(x)$ を用いた。
同様に,$C_2(x)$ だけの式を作って積分すると

\begin{gather}
  C'_2(x) = \frac{y_1(x)r(x)}{W(x)}
  \Rightarrow C_2(x) =
  \int_{x_2}^{x}\frac{y_1(\hat{x})r(\hat{x})}{W(\hat{x})}\:d\hat{x} \label{33}
\end{gather}

先ほどと同様,$x_2$ は任意の定数である。
式 \eqref{32}, \eqref{33} を最初に仮定した特解の表式 \eqref{25} に代入して,公式 \eqref{23} を得る。


% 収束半径
\newpage
\section{収束半径}
\begin{itembox}[l]{ダランベールの収束判定法}
  数列 $\{a_{n}\}$に対し、
  \begin{gather}
    \lim_{n\to\infty}\left|\frac{a_{n+1}}{a_n}\right| = r
  \end{gather}
  が存在するとする。このとき,\dm{\sum_{n=0}^{\infty}a_{n}}
  の収束・発散について
  \begin{enumerate}[label=\cdot]
    \item $0\leq r<1$ ならば絶対収束
    \item $1<r$ ならば発散
  \end{enumerate}
  となる。
\end{itembox}

\begin{itembox}[l]{コーシーアダマールの収束判定法}
  数列 $\{a_{n}\}$に対し、
  \begin{gather}
    \lim_{n\to\infty}\sup \sqrt[n]{|a_{n}|} = r
  \end{gather}
  が存在するとする。このとき,\dm{\sum_{n=0}^{\infty}a_{n}}
  の収束・発散について
  \begin{enumerate}[label=\cdot]
    \item $0\leq r<1$ ならば絶対収束
    \item $1<r$ ならば発散
  \end{enumerate}
  となる。
\end{itembox}



この議論より一般のべき級数展開に対しても、
$x$についての収束半径を得る。

ある関数$f(x)$のべき級数展開を考える。
\begin{gather}
  f(x) = a_0 + a_1 x + a_2 x^2 + \cdots + a_n x^n + \cdots
\end{gather}
この関数が収束する条件を考える。

マクローリン展開の剰余項が実際に無視できて項数が無限大の
展開された関数が元の関数と一致することを示す。


% 極値
\newpage
\section{極値とヘッセ行列とヘッシアン(Hessian)}
ヘッセ行列$\bH$の行列式$\det\bH=H$の値から

\begin{enumerate}[label=\arabic*.]
  \item $\bH\succ 0$\:\eqa\: $H>0$ \land\, $f_{xx}>0$\Rightarrow 極小値
  \item $\bH\prec 0$\:\eqa\: $H>0$ \land\, $f_{xx}<0$\Rightarrow 極大値
  \item $H=0$ \Rightarrow 不明。個別に計算する % notice: https://detail.chiebukuro.yahoo.co.jp/qa/question_detail/q1092575770
  \item $H<0$ \Rightarrow 極値ではない
\end{enumerate}

行列$\bH\succ 0$は正定値、$\bH\succeq 0$は半正定値、
$\bH\prec 0$は負定値、$\bH\preceq 0$は半負定値といい、
次\footnote{梅谷俊治『しっかり学ぶ数理最適化』pp.90-91 など}のように定義される。
\begin{itembox}[l]{行列の符号}
  $n$次正方行列$\bA\in\bbR^{n\tm n}$が任意の$\bx\in\bbR^{n}$に対して
  \begin{gather}
    \bx^{\top}\bA\bx = \sum_{i=1}^{n}a_{ii}x_{i}^2 + \sum_{i\neq j}^{}a_{ij}x_{i}x_{j}\geq 0
  \end{gather}
  を満たすとき、行列$\bA$は半正定値(positive semidefinite)であると呼ぶ。また、任意の$\bx\in\bbR^{n}$に対して
  $\bx^{\top}\bA\bx>0$を満たすとき、行列$\bA$は正定値(positive definite)であると呼ぶ。

  また、
  \begin{align}
    \text{行列$\bA$が(半)正定値}
    &\eqa \bA\succ(\succeq) 0 \\
    &\eqa {\forall \bx\in\bbR^{n},\,\bx^{\top}\bA\bx>(\geq) 0} \\
    &\eqa \text{行列$\bA$の固有値が正(非負)} \\
    &\eqa \text{すべての主小行列式が正(主小行列式が非負)} \\
    &\eqa \text{すべての首座小行列式が正}
  \end{align}
  である。
  ただし、行列$\bA = [a_{ij}]\in\bbR^{n\tm n}\:(i,j\in N,\,N=\{1,2,\cdots,N\})$
  に対して($k$次)首座小行列$\hat{\bA}$とは
  $k(\in\bbZ,\,0\leq k\leq n)$とある集合$K=\{1,2,\cdots,k\}$に対して
  \begin{gather}
    \hat{\bA} = [a_{ij}] \quad(i,j\in K)
  \end{gather}
  を満たす行列である。
  また、主小行列とは$M\subseteq N$に対して
  \begin{gather}
    \tilde{\bA} = [a_{ij}]\quad(i,j\in M)
  \end{gather}
  を満たす行列である。

  $-\bA$が(半)正定値であるとき、$\bA$は(半)負定値であるという。
  また、正定値、半正定値、負定値、半負定値のいずれでもないものを不定値行列という。
\end{itembox}


% ワイエルシュトラス変換
\newpage
\section{ワイエルシュトラス(Weierstrass)変換}
三角関数の有理式の積分について、
変数変換(置換)によって積分を求めるもの。
\begin{screen}
  \dm{t = \tan \frac{\grt}{2}}のとき、
  \begin{gather}
    \cos \grt = \frac{1-t^2}{1+t^2} \\
    \sin \grt = \frac{2t}{1+t^2} \\
    \tan \grt = \frac{2t}{1-t^2} \\
    d\grt = \frac{2}{1+t^2}dt
  \end{gather}
  として変数変換できる。
\end{screen}


% =============================================================
% 線形代数
% - - - - - - - - - - - - -
\newpage
% 対角化
\newpage
\section{対角化}
$n\times n$の行列$\bA$の対角化は 
\begin{enumerate}[label=\arabic*.]
  \item 固有値が$n$個
  \begin{fleqn}[10pt]
    \begin{align*}
      \Rightarrow\text{可能}
    \end{align*}
  \end{fleqn}
  \item 固有値1個($n$重解)
  \begin{fleqn}[10pt]
    \begin{align*}
      \Rightarrow
      \begin{dcases*}
        \text{可能} & ($\lambda\bE-\bA = \bm{0}$)\\
        \text{不可能} & (o/w)
      \end{dcases*}
    \end{align*}
  \end{fleqn}
  \item 固有値に$m$重解を含む
  \begin{fleqn}[10pt]
    \begin{align*}
      \Rightarrow
      \begin{dcases*}
        \text{可能} & ($m$重解で固有ベクトルが$m$個対応する場合)\\
        \text{不可能} & (o/w:この問題はこれに該当)
      \end{dcases*}
    \end{align*}
  \end{fleqn}
\end{enumerate}


% 2次形式と標準形
2次形式と一般形



\newpage
% 参考文献
\begin{thebibliography}{99}
  \bibitem{ref:teisuhenkaho}『二階非斉次常微分方程式の解法』\url{https://www2.yukawa.kyoto-u.ac.jp/~norihiro.tanahashi/pdf/ODE/note_7.pdf}
  % \bibitem{ref:positive-definite}梅谷俊治『しっかり学ぶ数理最適化』
\end{thebibliography}

\end{document}
