\renewcommand{\figurename}{図}
\renewcommand{\thefigure}{\arabic{figure}:}
\renewcommand{\tablename}{表}
\renewcommand{\thetable}{\arabic{table}:}

% 再帰的にコマンドを改変する
% \let\origincaption\caption
% \renewcommand{\caption}[1]{\origincaption{#1}}

% -------------------------------------------------------%

% \siを使うと数字と単位の間が空かないので致し方なく\siを書き変えてる
\let\originsi\si
\renewcommand{\si}[1]{\hspace*{1pt}\originsi{#1}}

% % 連分数が少し詰まるのを若干改善と思ったけどあんまりよくないのでやめました
% \let\origincfrac\cfrac
% \renewcommand{\cfrac}[2]{%
%   \raisebox{1pt}{$\displaystyle\origincfrac{#1}{\raisebox{.5pt}{$#2$}}$}%
% }


% -------------------------------------------------------%

\makeatletter

\newcommand{\reference}[0]{\setlength{\hangindent}{18pt}\noindent}

% -------------------------------------------------------%

\newcommand{\yearn}[0]{\number\year}
\newcommand{\monthn}[0]{\number\month}
\newcommand{\dayn}[0]{\number\day}

% -------------------------------------------------------%

% 数式や図表のref
\renewcommand{\refeq}[1]{\eqref{#1}式}
\newcommand{\reffig}[1]{図\ref{#1}}
\newcommand{\reftbl}[1]{表\ref{#1}}

% -------------------------------------------------------%

% 図挿入(作ったけど使わず)
\newcommand{\insfig}[2][.85]{%
  \begin{figure}[H]\centering
    \includegraphics[width=#1\linewidth]{#2}
  \end{figure}
}


% -------------------------------------------------------%

% 雑多なやつら
% 「$x$軸、平面、座標」のように書くとスペースが空くので
% \dspaceと入れてほんのすこーーしだけ詰める
% \newcommand{\axis}{\hspace*{-0.05pt}\,軸}
% \newcommand{\pln}{\hspace*{-0.05pt}\,面}
\newcommand{\dspace}{\hspace*{-0.05pt}\,}

% -------------------------------------------------------%

% よく使う単位
\newcommand{\degC}[0]{\mathrm{{}^\circ \hspace*{-0.5pt} C}}
\renewcommand{\deg}[0]{\mathrm{{}^\circ}}

% -------------------------------------------------------%

% よく使う演算子
\renewcommand{\Vector}[1]{{\mbox{\boldmath$#1$}}}
\newcommand{\tensor}[1]{\undertilde{#1}}
\renewcommand{\rm}[1]{\mathrm{#1}}

% -------------------------------------------------------%

% displaystyle math mode
\newcommand{\dm}[1]{
  $\displaystyle #1 $
}
\newcommand{\disp}{\displaystyle}

% -------------------------------------------------------%

% よく使う記号

% =================================
% 太字行列・ベクトル
\newcommand{\bA}{\bm{A}}
\newcommand{\bB}{\bm{B}}
\newcommand{\bE}{\bm{E}}
\newcommand{\bC}{\bm{C}}
\newcommand{\bD}{\bm{D}}
\newcommand{\bH}{\bm{H}}
\newcommand{\bI}{\bm{I}}
\newcommand{\bL}{\bm{L}}
\newcommand{\bU}{\bm{U}}
\newcommand{\bP}{\bm{P}}
\newcommand{\bQ}{\bm{Q}}

\newcommand{\bbR}{\mathbb{R}}
\newcommand{\bbC}{\mathbb{C}}
\newcommand{\bbN}{\mathbb{N}}
\newcommand{\bbZ}{\mathbb{Z}}

\newcommand{\ba}{{\bm{a}}}
\newcommand{\bb}{{\bm{b}}}
\newcommand{\bc}{{\bm{c}}}
\newcommand{\bd}{{\bm{d}}}
\newcommand{\be}{{\bm{e}}}
\newcommand{\bg}{{\bm{g}}}

\newcommand{\bn}{{\bm{n}}}

\newcommand{\bp}{{\bm{p}}}

\newcommand{\bt}{{\bm{t}}}

\newcommand{\bx}{{\bm{x}}}
\newcommand{\by}{{\bm{y}}}
\newcommand{\bz}{{\bm{z}}}

\newcommand{\bu}{{\bm{u}}}
\newcommand{\bv}{{\bm{v}}}
\newcommand{\bw}{{\bm{w}}}

% bold zero vector
\newcommand{\bzv}{\bm{0}}

% =================================
% ギリシャ文字
\newcommand{\ve}{\varepsilon}
\newcommand{\vp}{\varphi}

\newcommand{\gra}{{\alpha}}
\newcommand{\grg}{{\gamma}}
\newcommand{\grd}{{\delta}}
\newcommand{\grt}{{\theta}}
\newcommand{\grk}{{\kappa}}
\newcommand{\grl}{{\lambda}}
\newcommand{\grs}{{\sigma}}
\newcommand{\gro}{{\omega}}
\newcommand{\grp}{{\phi}}


\newcommand{\grG}{{\Gamma}}
\newcommand{\grL}{{\Lambda}}

% =================================
% 略記号
\newcommand{\tm}{\times}
\newcommand{\lra}{\longrightarrow}
\newcommand{\eqa}{\Leftrightarrow} % equivalent arrowのつもり

% -------------------------------------------------------%


% -------------------------------------------------------%

% 数式番号にセクション番号を併記する
\renewcommand{\theequation}{\thesection.\arabic{equation}}
\makeatletter
\@addtoreset{equation}{section}
\makeatother

% -------------------------------------------------------%

% 問題文の左の線の定義
\renewenvironment{leftbar}{%
\def\FrameCommand{\hspace{10pt}\vrule width 1.2pt \hspace{10pt}}%
\MakeFramed {\advance\hsize-\width \FrameRestore}}%
{\endMakeFramed}

% subsubsectionを太字の「問1」表示にする
\renewcommand{\thesubsubsection}{\large\textbf{問\arabic{subsubsection}}}

% -------------------------------------------------------%

% 定理スタイルの定義
\newtheoremstyle{mystyle}
  {\topsep}   % スペース上
  {\topsep}   % スペース下
  {\normalfont}  % 本文のフォント
  {0pt}       % インデント
  {\bfseries} % タイトルのフォント
  {.}         % 句読点
  {.5em}      % タイトルと本文のスペース
  {}          % タイトルのスペース
% 定理環境の作成
\theoremstyle{mystyle}
\newtheorem*{ans*}{解答}
\newtheorem*{other*}{別解}
\newtheorem*{supple*}{Supplement}
\newtheorem*{append*}{Appendix}

% -------------------------------------------------------%

% オペレーター
\DeclareMathOperator{\rank}{rank}
\DeclareMathOperator{\Ker}{Ker}
% \DeclareMathOperator{\Im}{Im}
% 虚部の「ℑ」を更新(像として使いたい)
\renewcommand{\Im}{\operatorname{Im}}
\DeclareMathOperator{\proj}{proj}

% -------------------------------------------------------%

% 二項演算子
\renewcommand{\parallel}{/\!/}

% 床関数(ガウス関数)
\newcommand{\flr}[1]{\lfloor #1 \rfloor} % ふつーの床関数
\newcommand{\gflr}[1]{\left[ #1 \right]} % ガウス記号を使った床関数

% -------------------------------------------------------%

% 演算子
\newcommand{\ppar}[2]{\frac{\partial #1}{\partial #2}}
\renewcommand{\d}{\partial}
\newcommand{\pd}{\partial}

% -------------------------------------------------------%
% 問題(problem)
% セクションやサブセクションのあと以外は改ページする
\newcommand{\prob}[1]{%
  \clearpageornot
  \subsubsection{}\begin{leftbar}#1\end{leftbar}%
  \setcounter{cpornot}{0}
}
% 小論文
\newcommand{\essayprob}[1]{%
  % \clearpageornot
  \begin{leftbar}#1\end{leftbar}
  \setcounter{cpornot}{0}
}
% 専門科目問題(specialized problem)
\newcommand{\spprob}[1]{%
  \begin{leftbar}#1\end{leftbar}%
}

% -------------------------------------------------------%

% -------------------------------------------------------%

% alignを書くのが面倒なとき
\newcommand{\al}[1]{\begin{align}#1\end{align}}
% ベクトル
\renewcommand{\vec}[1]{\begin{Bmatrix}#1\end{Bmatrix}}
% 行列
\newcommand{\bmat}[1]{\begin{bmatrix}#1\end{bmatrix}} % [A]
\newcommand{\Bmat}[1]{\begin{Bmatrix}#1\end{Bmatrix}} % {A}
\newcommand{\pmat}[1]{\begin{pmatrix}#1\end{pmatrix}} % (A)
\newcommand{\vmat}[1]{\begin{vmatrix}#1\end{vmatrix}} % |A|

% ロピタル
\newcommand{\lhopital}{L'H\^{o}pital}
% 集合
\newcommand{\Dset}[1]{\left\{ #1 \right\}}
% 集合の{(x,y)|x<0}みたいに書くときの縦線
\newcommand{\relmiddle}{\mathrel{}\middle| \mathrel{}}

% 内積
\newcommand{\ip}[1]{\langle#1\rangle}
