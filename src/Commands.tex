\renewcommand{\figurename}{図}
\renewcommand{\thefigure}{\arabic{figure}:}
\renewcommand{\tablename}{表}
\renewcommand{\thetable}{\arabic{table}:}

% 再帰的にコマンドを改変する
% \let\origincaption\caption
% \renewcommand{\caption}[1]{\origincaption{#1}}

% -------------------------------------------------------%

% \siを使うと数字と単位の間が空かないので致し方なく\siを書き変えてる
\let\originsi\si
\renewcommand{\si}[1]{\hspace*{1pt}\originsi{#1}}

% -------------------------------------------------------%

\makeatletter

\newcommand{\reference}[0]{\setlength{\hangindent}{18pt}\noindent}

% -------------------------------------------------------%

\newcommand{\yearn}[0]{\number\year}
\newcommand{\monthn}[0]{\number\month}
\newcommand{\dayn}[0]{\number\day}

% -------------------------------------------------------%

% 数式や図表のref
\renewcommand{\refeq}[1]{\eqref{#1}式}
\newcommand{\reffig}[1]{図\ref{#1}}
\newcommand{\reftbl}[1]{表\ref{#1}}

% -------------------------------------------------------%

% 図挿入(作ったけど使わず)
\newcommand{\insfig}[2][.85]{%
  \begin{figure}[H]\centering
    \includegraphics[width=#1\linewidth]{#2}
  \end{figure}
}


% -------------------------------------------------------%

% 雑多なやつら
% 「$x$軸、平面、座標」のように書くとスペースが空くので
% \dspaceと入れてほんのすこーーしだけ詰める
% \newcommand{\axis}{\hspace*{-0.05pt}\,軸}
% \newcommand{\pln}{\hspace*{-0.05pt}\,面}
\newcommand{\dspace}{\hspace*{-0.05pt}\,}

% -------------------------------------------------------%

% よく使う単位
\newcommand{\degC}[0]{\mathrm{{}^\circ \hspace*{-0.5pt} C}}
\renewcommand{\deg}[0]{\mathrm{{}^\circ}}

% -------------------------------------------------------%

% よく使う演算子
\renewcommand{\Vector}[1]{{\mbox{\boldmath$#1$}}}
\newcommand{\tensor}[1]{\undertilde{#1}}
\renewcommand{\rm}[1]{\mathrm{#1}}

% -------------------------------------------------------%

% displaystyle math mode
\newcommand{\dm}[1]{
  $\displaystyle #1 $
}
\newcommand{\disp}{\displaystyle}

% -------------------------------------------------------%

% よく使う記号

% =================================
% 太字行列・ベクトル
\newcommand{\bA}{\bm{A}}
\newcommand{\bB}{\bm{B}}
\newcommand{\bE}{\bm{E}}
\newcommand{\bC}{\bm{C}}
\newcommand{\bH}{\bm{H}}
\newcommand{\bL}{\bm{L}}
\newcommand{\bU}{\bm{U}}
\newcommand{\bP}{\bm{P}}

\newcommand{\ba}{{\bm{a}}}
\newcommand{\bb}{{\bm{b}}}
\newcommand{\bc}{{\bm{c}}}
\newcommand{\bd}{{\bm{d}}}
\newcommand{\be}{{\bm{e}}}

\newcommand{\bn}{{\bm{n}}}

\newcommand{\bx}{{\bm{x}}}
\newcommand{\by}{{\bm{y}}}

\newcommand{\bu}{{\bm{u}}}
\newcommand{\bv}{{\bm{v}}}
% \newcommand{\zerov}{\bm{0}}

% =================================
% ギリシャ文字
\newcommand{\ve}{\varepsilon}
\newcommand{\vp}{\varphi}

\newcommand{\gra}{{\alpha}}
\newcommand{\grg}{{\gamma}}
\newcommand{\grd}{{\delta}}
\newcommand{\grt}{{\theta}}
\newcommand{\grk}{{\kappa}}
\newcommand{\grl}{{\lambda}}
\newcommand{\grs}{{\sigma}}
\newcommand{\gro}{{\omega}}

% =================================
% 略記号
\newcommand{\tm}{\times}


% -------------------------------------------------------%


% -------------------------------------------------------%

% 数式番号にセクション番号を併記する
\renewcommand{\theequation}{\thesection.\arabic{equation}}
\makeatletter
\@addtoreset{equation}{section} 
\makeatother

% -------------------------------------------------------%

% 問題文の左の線の定義
\renewenvironment{leftbar}{%
\def\FrameCommand{\hspace{10pt}\vrule width 1.2pt \hspace{10pt}}%
\MakeFramed {\advance\hsize-\width \FrameRestore}}%
{\endMakeFramed}

% subsubsectionを太字の「問1」表示にする
\renewcommand{\thesubsubsection}{\large\textbf{問\arabic{subsubsection}}}

% -------------------------------------------------------%

% 定理スタイルの定義
\newtheoremstyle{mystyle}
  {\topsep}   % スペース上
  {\topsep}   % スペース下
  {\normalfont}  % 本文のフォント
  {0pt}       % インデント
  {\bfseries} % タイトルのフォント
  {.}         % 句読点
  {.5em}      % タイトルと本文のスペース
  {}          % タイトルのスペース
% 定理環境の作成
\theoremstyle{mystyle}
\newtheorem*{ans*}{解答}
\newtheorem*{other*}{別解}
\newtheorem*{supple*}{Supplement}
\newtheorem*{append*}{Appendix}

% -------------------------------------------------------%

% 数学文字表現
\DeclareMathOperator{\rank}{rank}


% -------------------------------------------------------%

% 数学記号
\renewcommand{\parallel}{/\!/}

% -------------------------------------------------------%

% 演算子
\newcommand{\ppar}[2]{\frac{\partial #1}{\partial #2}}
\renewcommand{\d}{\partial}

% -------------------------------------------------------%
% 問題(problem)
\newcommand{\prob}[1]{\subsubsection{}\begin{leftbar}\noindent#1\end{leftbar}}

% -------------------------------------------------------%
% 小論文
\newcommand{\essayprob}[1]{\begin{leftbar}#1\end{leftbar}}

% -------------------------------------------------------%
% 専門科目問題(specialized problem)
\newcommand{\spprob}[1]{\begin{leftbar}#1\end{leftbar}}

% alignを書くのが面倒なとき
\newcommand{\al}[1]{\begin{align}#1\end{align}}

% 集合の{(x,y)|x<0}みたいに書くときの縦線
\newcommand{\relmiddle}{\mathrel{}\middle| \mathrel{}}
