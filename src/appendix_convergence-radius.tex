\newpage
\section{収束半径}
\begin{itembox}[l]{ダランベールの収束判定法}
  数列 $\{a_{n}\}$に対し、
  \begin{gather}
    \lim_{n\to\infty}\left|\frac{a_{n+1}}{a_n}\right| = r
  \end{gather}
  が存在するとする。このとき,\dm{\sum_{n=0}^{\infty}a_{n}}
  の収束・発散について
  \begin{enumerate}[label=\cdot]
    \item $0\leq r<1$ ならば絶対収束
    \item $1<r$ ならば発散
  \end{enumerate}
  となる。
\end{itembox}

\begin{itembox}[l]{コーシーアダマールの収束判定法}
  数列 $\{a_{n}\}$に対し、
  \begin{gather}
    \lim_{n\to\infty}\sup \sqrt[n]{|a_{n}|} = r
  \end{gather}
  が存在するとする。このとき,\dm{\sum_{n=0}^{\infty}a_{n}}
  の収束・発散について
  \begin{enumerate}[label=\cdot]
    \item $0\leq r<1$ ならば絶対収束
    \item $1<r$ ならば発散
  \end{enumerate}
  となる。
\end{itembox}



この議論より一般のべき級数展開に対しても、
$x$についての収束半径を得る。

ある関数$f(x)$のべき級数展開を考える。
\begin{gather}
  f(x) = a_0 + a_1 x + a_2 x^2 + \cdots + a_n x^n + \cdots
\end{gather}
この関数が収束する条件を考える。

マクローリン展開の剰余項が実際に無視できて項数が無限大の
展開された関数が元の関数と一致することを示す。
