\section{2022 春}
\essayprob{
  近年、全国各地で水災害が激甚化・頻発化するとともに、
  気候変動の影響により、今後、降雨量や洪水発生頻度が全国で増加することが見込まれている。
  そこで、治水対策として「氾濫をできるだけ防ぐための対策」に加え、
  「被害対象を減少させるための対策」や「被害の軽減、早期復旧、復興のための対策」
  へと大きな転換が図られ、国や流域自治体、企業・住民等、あらゆる関係者が協働して取り組む
  「流域治水」の実効性を高める「流域治水関連法」が2021年11月より施行された。

  あなたが実施すべきと考える流域治水事業の具体策を提案し、
  その意義と実行にあたっての課題を論ぜよ(1200字以内)。
}
