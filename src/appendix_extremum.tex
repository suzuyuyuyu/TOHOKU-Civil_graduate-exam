\newpage
\section{極値とヘッセ行列とヘッシアン(Hessian)}
ヘッセ行列$\bH$の行列式$\det\bH=H$の値から

\begin{enumerate}[label=\arabic*.]
  \item $\bH\succ 0$\:\eqa\: $H>0$ \land\, $f_{xx}>0$\Rightarrow 極小値
  \item $\bH\prec 0$\:\eqa\: $H>0$ \land\, $f_{xx}<0$\Rightarrow 極大値
  \item $H=0$ \Rightarrow 不明。個別に計算する % notice: https://detail.chiebukuro.yahoo.co.jp/qa/question_detail/q1092575770
  \item $H<0$ \Rightarrow 極値ではない
\end{enumerate}

行列$\bH\succ 0$は正定値、$\bH\succeq 0$は半正定値、
$\bH\prec 0$は負定値、$\bH\preceq 0$は半負定値といい、
次\footnote{梅谷俊治『しっかり学ぶ数理最適化』pp.90-91 など}のように定義される。
\begin{itembox}[l]{行列の符号}
  $n$次正方行列$\bA\in\bbR^{n\tm n}$が任意の$\bx\in\bbR^{n}$に対して
  \begin{gather}
    \bx^{\top}\bA\bx = \sum_{i=1}^{n}a_{ii}x_{i}^2 + \sum_{i\neq j}^{}a_{ij}x_{i}x_{j}\geq 0
  \end{gather}
  を満たすとき、行列$\bA$は半正定値(positive semidefinite)であると呼ぶ。また、任意の$\bx\in\bbR^{n}$に対して
  $\bx^{\top}\bA\bx>0$を満たすとき、行列$\bA$は正定値(positive definite)であると呼ぶ。

  また、
  \begin{align}
    \text{行列$\bA$が(半)正定値}
    &\eqa \bA\succ(\succeq) 0 \\
    &\eqa {\forall \bx\in\bbR^{n},\,\bx^{\top}\bA\bx>(\geq) 0} \\
    &\eqa \text{行列$\bA$の固有値が正(非負)} \\
    &\eqa \text{すべての主小行列式が正(主小行列式が非負)} \\
    &\eqa \text{すべての首座小行列式が正}
  \end{align}
  である。
  ただし、行列$\bA = [a_{ij}]\in\bbR^{n\tm n}\:(i,j\in N,\,N=\{1,2,\cdots,N\})$
  に対して($k$次)首座小行列$\hat{\bA}$とは
  $k(\in\bbZ,\,0\leq k\leq n)$とある集合$K=\{1,2,\cdots,k\}$に対して
  \begin{gather}
    \hat{\bA} = [a_{ij}] \quad(i,j\in K)
  \end{gather}
  を満たす行列である。
  また、主小行列とは$M\subseteq N$に対して
  \begin{gather}
    \tilde{\bA} = [a_{ij}]\quad(i,j\in M)
  \end{gather}
  を満たす行列である。

  $-\bA$が(半)正定値であるとき、$\bA$は(半)負定値であるという。
  また、正定値、半正定値、負定値、半負定値のいずれでもないものを不定値行列という。
\end{itembox}
