\section{2022秋}

\setcounter{yearcounter}{2022}
% \setcounter{page}{1}


\subsection{微分積分}
\prob{%
  次の極限値を求めよ。
  \al{\lim_{x\to0}\frac{x^2\sin(1/x)}{\sin x}}
}
\begin{ans*}
  ${}$
  \begin{gather}
    \lim_{x\to 0}\frac{x^2\sin\cfrac{1}{x}}{\sin x}
    = \lim_{x\to 0}\left(\frac{x}{\sin x}\cdot x\sin\frac{1}{x}\right)
  \end{gather}
  \dm{-1\leq\sin\frac{1}{x}\leq 1}より
  \begin{gather}
    \begin{dcases}
      -\frac{x}{\sin x}\cdot x
      \leq \frac{x}{\sin x}\cdot x\sin\frac{1}{x}
      \leq \frac{x}{\sin x}\cdot x
      & (\text{if\:\:} \frac{x^2}{\sin x} > 0) \\
      \frac{x}{\sin x}\cdot x
      \leq \frac{x}{\sin x}\cdot x\sin\frac{1}{x}
      \leq -\frac{x}{\sin x}\cdot x
      & (\text{if\:\:} \frac{x^2}{\sin x} < 0)
    \end{dcases}
  \end{gather}
  いずれについても\dm{\lim_{x\to 0}\frac{x}{\sin x} = 1}より
  最左辺、最右辺$\longrightarrow 0\:(x\to 0)$であるので
  はさみうちの原理から
  \begin{gather}
    \lim_{x\to 0}\frac{x^2 \sin \cfrac{1}{x}}{\sin x} = 0
  \end{gather}
\end{ans*}


\prob{%
  次の関数$f(x,y)$の極値を求めよ。
  \begin{gather}
    f(x,y) = xy + \frac{8}{x} + \frac{1}{y} \quad(x\neq 0,\, y\neq 0)
  \end{gather}
}
\begin{ans*}
  停留点であるためには
  \begin{align}
    &\begin{dcases*}
      \ppar{f}{x} = y - \frac{8}{x^2} = 0 \\
      \ppar{f}{y} = x - \frac{1}{y^2} = 0
    \end{dcases*}\\
    &\Rightarrow
    \begin{dcases*}
      x = 4 \\
      y = \frac{1}{2}
    \end{dcases*}
  \end{align}
  かつ、この点でヘッシアン\footnote{appendix}$\det\bH$は
  \begin{align}
    \det\bH
    &=
    \left.
    \det
    \bmat{
    \disp\frac{\d^2f}{\d x^2} & \disp\frac{\d^2f}{\d x\d y} \\
    \\
    \disp\frac{\d^2f}{\d x\d y} & \disp\frac{\d^2f}{\d y^2}
    }\right|_{(x,y) = (4,1/2)} \\
    &=
    \begin{vmatrix}
      1/4 & 1 \\
      1 & 16
    \end{vmatrix}
    = 3 > 0
  \end{align}
  より\dm{\bH \succ 0}なので\dm{(x,y) = (4,1/2)}で極小値6をとる。
\end{ans*}

\prob{%
  次に示す$xyz$空間の領域$D$を考える。
  このとき、以下の問いに答えよ。
  \begin{gather}
    D = \Dset{(x,y,z)\relmiddle 0\leq z\leq x^2+y^2, x^2+y^2\leq 2x}
  \end{gather}

  \begin{enumerate}[label=(\arabic*)]
    \item 領域$D$を図示せよ。
    \item 領域$D$の体積を求めよ。
  \end{enumerate}
}
\begin{ans*}
  ${}$
  \begin{enumerate}[label=(\arabic*)]
    \item 略 % pic 図示
    \item ある座標$(x,y)$における領域の上限は$x^2+y^2=z$上の点であるので、
    高さを$z=x^2 + y^2$、積分領域を\dm{C = \Dset{(x,y)\relmiddle (x-1)^2+y^2 \leq 1}}として
    求める体積$V_{\rm{D}}$は
    \begin{gather}
      V_{\rm{D}} = \iint_C z \,dxdy
    \end{gather}
    と求められる。
    まず、この積分領域は極座標系$(r,\theta)$に変換すれば
    \begin{align}
      C
      &= \Dset{(x,y)\relmiddle (x-1)^2 + y^2 \leq 1} \\
      &= \Dset{(r,\theta)\relmiddle r^2 - 2r\cos\theta\leq 1} \\
      &= \Dset{(r,\theta)\relmiddle 0\leq r\leq 2\cos\theta, 0\leq \theta < 2\pi}
    \end{align}
    であるので、
    \begin{align}
      V_{\rm{D}}
      &= \int_{0}^{2\pi}\int_{0}^{2\cos\theta}r^3\,drd\theta \\
      &= \int_{0}^{2\pi} 4\cos^4\theta\,d\theta \\
      &= \int_{0}^{2\pi} \left(1 + 2\cos\theta + \frac{1 + \cos4\theta}{2}\right)\,d\theta \\
      &= 3\pi
    \end{align}
  \end{enumerate}

\end{ans*}

\prob{%
  次の微分方程式を解け。
  ここで$e$は自然対数の底である。
  \begin{gather}
    y'' + 4y' + 4y - e^{-2x} = 0
  \end{gather}
}

\begin{ans*}
  与えられた微分方程式の斉次形である$y'' + 4y' + 4y = 0$について、
  特性方程式より
  \begin{gather}
    \grl^2+4\grl+4=0\eqa \grl = -2
  \end{gather}
  斉次解は\dm{y = (A+Bx)e^{-2x}}

  また、特解を\dm{y_{\rm{p}} = Cx^2e^{-2x}}とおく\footnote{appendix}と
  \begin{gather}
    \begin{dcases*}
      y'_{\rm{p}} = 2C(x-x^2)e^{-2x} \\
      y''_{\rm{p}}= 2C(2x^2-4x+1)e^{-2x}
    \end{dcases*}
  \end{gather}
  であるので与えられた微分方程式に代入して
  \begin{gather}
    2C(2x^2 - 4x + 1) + 8C(x-x^2) + 4Cx^2 = 1 \\
    C = \frac{1}{2}
  \end{gather}
  よって、求める一般解は斉次解と特解の足し合わせで
  \begin{gather}
    y = \left(A + Bx + \frac{1}{2}x^2\right)e^{-2x}
  \end{gather}
\end{ans*}


\newpage
\subsection{線形代数}
\prob{%
  以下の行列の行列式を求めよ。
  \al{
    \pmat{
      0 & 1 & 2 & 0 \\
      -4 & 0 & 3 & 1 \\
      3 & -3 & 0 & 0 \\
      6 & 0 & -2 & 0 \\
    }
  }
}
\begin{ans*}
  \begin{align}
    \det
    \bmat{
      0 & 1 & 2 & 0 \\
      -4 & 0 & 3 & 1 \\
      3 & -3 & 0 & 0 \\
      6 & 0 & -2 & 0 \\
    }
    = \det
    \bmat{
      0  & 1  & 2  \\
      3  & -3 & 0  \\
      6  & 0  & -2 \\
    }
    = 42
  \end{align}
\end{ans*}

\prob{%
  以下の行列
  \begin{gather}
    \bA =
    \pmat{
      {\disp \frac{13}{2}} & {\disp \frac{5}{2}} \\
      \\
      {\disp \frac{5}{2}} & {\disp \frac{13}{2}} \\
    }
  \end{gather}
  は固有値$\grl_1 = 4, \grl_2 = 9$を持つ。
  この行列について以下の問いに答えよ。

  \begin{enumerate}[label=(\arabic*)]
    \item \bA の正規化された固有ベクトルをすべて求めよ。
    \item $\bB^2 = \bA$となる行列$\bB$を求めよ。
  \end{enumerate}
}
\begin{ans*}
  ${}$
  \begin{enumerate}[label=(\arabic*)]
    \item 固有ベクトルを$\bu$で表す。

    \begin{enumerate}[label=(\roman*)]
      \item $\grl = \grl_1$のとき
      \begin{gather}
        (\grl \bE - \bA)\bu
        =
        \bmat{
          {\disp -\frac{5}{2}} & {\disp -\frac{5}{2}} \\
          \\
          {\disp -\frac{5}{2}} & {\disp -\frac{5}{2}} \\
        }
        = \bm{0} \\
        \bu_1 =
        \frac{1}{\sqrt{2}}
        \Bmat{
          -1 \\ 1
        }
      \end{gather}
      \item $\grl = \grl_2$のとき
      \begin{gather}
        (\grl \bE - \bA)\bu
        =
        \bmat{
          {\disp \frac{5}{2}} & {\disp -\frac{5}{2}} \\
          \\
          {\disp -\frac{5}{2}} & {\disp \frac{5}{2}} \\
        }
        = \bm{0} \\
        \bu_2 =
        \frac{1}{\sqrt{2}}
        \Bmat{
          1 \\ 1
        }
      \end{gather}
    \end{enumerate}
    \item $\bB^2 = \bA$となるので$\bB$は$2\times2$の正方行列である。
    ここで、\dm{\bB = \bmat{a & b \\ c & d }}とおくと
    \begin{gather}
      \bB^2 =
      \bmat{
        a^2+bc & (a+d)b \\
        (a+d)b & bc+d^2
      }
      =
      \bmat{
        {\disp \frac{13}{2}} & {\disp \frac{5}{2}} \\
        \\
        {\disp \frac{5}{2}} & {\disp \frac{13}{2}} \\
      }
      \\
      \begin{dcases*}
        \frac{13}{2} = a^2 + bc \\
        \frac{5}{2} = (a+d)b \\
        \frac{5}{2} = (a+d)c \\
        \frac{13}{2} = bc + d^2
      \end{dcases*}
    \end{gather}
    整理して
    \begin{gather}
      a=d,b=c \\
      \begin{dcases*}
        \frac{13}{2} = a^2 + b^2 \\
        \frac{5}{2} = 2ab
      \end{dcases*} \\
      \therefore a + b = \pm 3,\, a - b = \pm 2\quad(\text{複号任意})
    \end{gather}
    を得る。よって、
    \begin{gather}
      \bB =
      \bmat{
        {\disp \pm\frac{1}{2}} & {\disp \pm\frac{5}{2}} \\
        \\
        {\disp \pm\frac{5}{2}} & {\disp \pm\frac{1}{2}} \\
      },\,
      \bmat{
        {\disp \pm\frac{5}{2}} & {\disp \pm\frac{1}{2}} \\
        \\
        {\disp \pm\frac{1}{2}} & {\disp \pm\frac{5}{2}} \\
      }\quad(\text{行列内は複号同順})
    \end{gather}
  \end{enumerate}
\end{ans*}


\prob{%
  $\Pi$ を以下のベクトルで張られる$\bbR^3$(三次元空間)の平面とする。
  \begin{gather}
    \ba_1 =
    \pmat{
      -1 \\ 0 \\ 1
    }
    ,\,
    \ba_2 =
    \pmat{
      1 \\ 1 \\ 0
    }
  \end{gather}
  以下の問いに答えよ。

  \begin{enumerate}[label=(\arabic*)]
    \item $\Pi$の正規直交基底$\bb_1,\bb_2$を求めよ。
    ただし$\ba_1 \parallel \bb_1$とする。
    \item $\{\bb_1, \bb_2, \bb_3\}$が$\bbR^3$の正規直交基底となるような$\bb_3$を求めよ。
  \end{enumerate}
}
\begin{ans*}
  ${}$
  \begin{enumerate}[label=(\arabic*)]
    % 線型独立な有限個のベクトルが与えられたとき、それらと同じ部分空間を張る正規直交系を作り出す
    \item Gram-Schmidtの直交化法を用いて得た2つのベクトルはもとのベクトルと
    同じ部分空間を張る。
    \begin{gather}
      \bb_1 = \frac{1}{\sqrt{2}}
      \Bmat{
      -1 \\ 0 \\ 1
      }
    \end{gather}
    $\ba_2$ の$\ba_1$への射影ベクトルは
    \begin{gather}
      (\ba_2\cdot\ba_1)\ba_1
    \end{gather}
    であるので、$\ba_2$はこの射影ベクトルと$\ba_1$と直交するベクトル$\bb'_2$を用いて
    \begin{gather}
      \ba_2 = (\ba_2\cdot\ba_1)\ba_1 + \bb'_2
    \end{gather}
    よって、$\bb_2$は
    \begin{align}
      \bb_2
      &=
      \frac{\bb'_2}{\|\bb'_2\|} \\
      &=
      \frac{1}{\sqrt{6}}
      \Bmat{
        1 \\ 2 \\ 1
      }
    \end{align}
    \item あらたな$\bb_3$は$\bb_1,\bb_2$の両者に直交するので外積として得られる。
    \begin{align}
      \bb_3
      &=
      \bb_1 \times \bb_2 \\
      &=
      \frac{1}{\sqrt{3}}
      \Bmat{
        -1 \\ 1 \\ -1
      }
    \end{align}
  \end{enumerate}
\end{ans*}

\begin{other*}
  $\bb_2,\bb_3$について、平面$\Pi$の法線ベクトル$\bn$から得ることを考える。法線ベクトルを
  \begin{gather}
    \bn =
    \Bmat{
      p \\ q \\ r
    }
  \end{gather}
  とおくと法線ベクトルは$\ba_1\cdot\bn=\ba_2\cdot\bn=0$であるので
  \begin{gather}
    \left\{
    \begin{aligned}
      -p + r &= 0 \\
      p + q &= 0
    \end{aligned}
    \right.
    \\
    \therefore \bn =
    \Bmat{
      1 \\ -1 \\ 1
    }
  \end{gather}
  $\bb_2\cdot\bn=\bb_2\cdot\bb_1=0$であるので、
  正規化されていない$\bb_2$と平行なベクトル$\bb'_2 = \{x, y, z\}^T$は
  \begin{gather}
    \left\{
    \begin{aligned}
      -x + z &= 0 \\
      x - y + z &= 0
    \end{aligned}
    \right.
    \\
    \therefore \bb'_2 =
    \Bmat{
      1 \\ 2 \\ 1
    }
  \end{gather}
  これを正規化して
  \begin{gather}
    \bb_2 =
    \frac{1}{\sqrt{6}}
    \Bmat{
      1 \\ 2 \\ 1
    }
  \end{gather}
  $\bb_3$は法線ベクトルと等しい方向で、
  \begin{align}
    \bb_3
    &= \frac{\bn}{|\bn|} \\
    &= \frac{1}{\sqrt{3}}
    \Bmat{
      1 \\ -1 \\ 1
    }
  \end{align}
\end{other*}
