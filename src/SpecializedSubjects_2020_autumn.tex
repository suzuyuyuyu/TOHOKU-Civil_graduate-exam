\newpage
\section{2020秋}
\setcounter{yearcounter}{2020}

\subsection{弾性体と構造の力学(1)}

Hooke則は以下のように与えられる。
\begin{align*}%\label{eq:}
  &\varepsilon_{xx} = \frac{1}{E}\{\sigma_{xx}-\nu(\sigma_{yy}+\sigma_{zz})\},
  \varepsilon_{yy} = \frac{1}{E}\{\sigma_{yy}-\nu(\sigma_{zz}+\sigma_{xx})\},
  \varepsilon_{zz} = \frac{1}{E}\{\sigma_{zz}-\nu(\sigma_{xx}+\sigma_{yy})\}\\
  &\gamma_{xy} = 2\varepsilon_{xy} = \frac{1}{G}\sigma_{xy},
  \gamma_{yz} = 2\varepsilon_{yz} = \frac{1}{G}\sigma_{yz},
  \gamma_{zx} = 2\varepsilon_{zx} = \frac{1}{G}\sigma_{zx}
\end{align*}

ここに、$E$はヤング率、$\nu$はポアソン比である。$G$はせん断弾性係であり、$E$と$\nu$を使って以下のように表される。

\begin{align*}%\label{eq:}
  G = \frac{E}{2(1+\nu)}
\end{align*}

線形弾性体に関する以下の問いに答えよ。

平面応力状態と平面応力ひずみ状態におけるひずみ成分$\varepsilon_{xx}$、$\varepsilon_{yy}$、$\gamma_{xy}$と
応力成分$\sigma_{xx}$、$\sigma_{yy}$、$\sigma_{xy}$の関係式を導出し、行列形式で表せ。

平面応力状態にある弾性版をx軸方向に一様な力で引っ張ったところ、以下に示す応力$\sigma$とひずみ$\varepsilon$
が発生した。板は初期に無応力状態にあり、均質一様に変形したものとする。この板のヤング率$E$とポアソン比$\nu$を求めよ。

\begin{align*}%\label{eq:}
  \begin{bmatrix}
    \sigma
  \end{bmatrix}
  =
  \begin{bmatrix}
    \sigma_{xx}&\sigma_{xy}\\
    \sigma_{xy}&\sigma_{yy}
  \end{bmatrix}
  =
  \begin{bmatrix}
    2.8&0\\
    0&0\\
  \end{bmatrix}
  (MPa),\:
  \begin{bmatrix}
    \varepsilon
  \end{bmatrix}
  =
  \begin{bmatrix}
    \varepsilon_{xx}&\varepsilon_{xy}\\
    \varepsilon_{xy}&\varepsilon_{yy}\\
  \end{bmatrix}
  =
  \begin{bmatrix}
    0.02&0\\
    0&-0.008
  \end{bmatrix}
\end{align*}

平面ひずみ状態にある弾性版に以下のひずみ$\varepsilon$を生じさせた。
弾性体の材質は設問2における板のと同じである。弾性体は初期に無応力状態にあり、
均質一様に変形したものとする。発生する応力の成分$\sigma_{xx}$、$\sigma_{yy}$、$\sigma_{xy}$を求めよ。

\begin{align*}%\label{eq:}
  \begin{bmatrix}
    \varepsilon
  \end{bmatrix}
  =
  \begin{bmatrix}
    \varepsilon_{xx}&\varepsilon_{xy}\\
    \varepsilon_{xy}&\varepsilon_{yy}
  \end{bmatrix}
  =
  \begin{bmatrix}
    0.01&0.02\\
    0.02&0.01
  \end{bmatrix}
\end{align*}

% [ ] 設問3における主応力$\sigma_1$、$\sigma_2$とそれらの主方向ベクトル$\{n_1\}=$

