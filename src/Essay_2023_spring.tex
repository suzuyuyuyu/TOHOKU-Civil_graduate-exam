\newpage
\section{2023春}
\setcounter{yearcounter}{2023}
\essayprob{%
  第5期科学技術基本計画で提唱されたSociety5.0では,現実空間(フィジカル空間)
  での膨大な観測データなどの情報を,仮想空間(サイバー空間)で分析して,
  その結果をより高度な社会を実現するために活用するというサイバー・フィジカルシステム
  (Cyber-physical System)が採用されている。

  土木工学において,現実空間では時間的・コスト的に不可能と言えるような様々な
  検討,設計,最適解探索を仮想空間で実行し,それを現実空間にフィードバックすることによって,
  効率化,産業創出,生産性向上等に活用できるような事例を挙げて,その意義,効果あるいは
  問題点について論ぜよ(1200字以内)。
}
