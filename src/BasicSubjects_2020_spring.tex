\newpage
\section{2020春}

\setcounter{yearcounter}{2020}
% \setcounter{page}{1}


\subsection{微分積分}
\prob{
  以下の問いに答えなさい。
  \begin{enumerate}[label=(\alph*)]
    \item 次の関数の微分を求めなさい。ただし、$a$は正の定数とする。
    \begin{gather}
      f(x) = a^x
    \end{gather}
    \item 次の関数の第$n$次導関数を求めなさい。
    \begin{gather}
      f(x) = e^x \sin x
    \end{gather}
  \end{enumerate}
}
% [?] (b)について、ライプニッツの定理を使っての解法があるかもしれません。

\begin{ans*}
  ${}$
  \begin{enumerate}[label=(\alph*)]
    \item
    \begin{gather}
      f'(x) = a^x \log a
    \end{gather}
    \item
    $f(x)$の$n$次導関数が
    \begin{gather}
      f^{(n)}(x) = (\sqrt{2})^n e^x\sin\biggl(x + \frac{n\pi}{4}\biggr) \label{eq:2020_spring_goalfnx}
    \end{gather}
    となることを数学的帰納法を用いて示す。
    \begin{enumerate}[label=(\roman*)]
      \item
      \begin{align}
        f'(x)
        &= e^x(\sin x + \cos x) \\
        &= \sqrt{2}e^x \sin\biggl(x + \frac{\pi}{4}\biggr) \\
        &= (\sqrt{2})^{1}e^x \sin\biggl(x + \frac{1\cdot\pi}{4}\biggr)
      \end{align}
      より$n=1$で\refeq{eq:2020_spring_goalfnx}は成り立つ。
      \item $n=k \,(\in \bbN)$で\refeq{eq:2020_spring_goalfnx}が成立すると仮定すると
      \begin{align}
        f^{(k)}(x) = (\sqrt{2})^k e^x\sin\biggl(x + \frac{k\pi}{4}\biggr)
      \end{align}
      であり、これを微分した$k+1$階の導関数は
      \begin{align}
        f^{(k+1)}(x)
        &= (\sqrt{2})^k e^x\biggl(\sin\biggl(x + \frac{k\pi}{4}\biggr) + \cos\biggl(x + \frac{k\pi}{4}\biggr)\biggr) \\
        &= (\sqrt{2})^{k+1} e^x\sin\biggl(x + \frac{k\pi}{4}+\frac{\pi}{4}\biggr) \\
        &= (\sqrt{2})^{k+1} e^x\sin\biggl(x + \frac{(k+1)\pi}{4}\biggr)
      \end{align}
      より$n=k+1$でも成り立つ。
    \end{enumerate}
    (i),(ii)と数学的帰納法より任意の自然数$n$に対して$n$次導関数が得られて、
    \begin{gather}
      f^{(n)}(x) = (\sqrt{2})^n e^x\sin\biggl(x + \frac{n\pi}{4}\biggr)
    \end{gather}
    である。
  \end{enumerate}

\end{ans*}
\prob{
  以下の式は閉曲線を極座標$(\rho,\phi)$で表示している。
  ただし、$a$と$b$は定数とする。
  \begin{gather}
    \frac{1}{\rho^2}
    = \frac{\cos^2\phi}{a^2} + \frac{\sin^2\phi}{b^2}
  \end{gather}
  \begin{enumerate}[label=(\alph*)]
    \item $x\text{-}y$座標系でこの式を表し、$y$について解きなさい。
    この曲線のグラフを描きなさい。
    \item 曲線内の面積を求めなさい。
  \end{enumerate}
}
\begin{ans*}
  ${}$
  \begin{enumerate}[label=(\alph*)]
    \item 両辺を$\rho$倍して、$x = \rho\cos\phi,\,y = \rho\sin\phi$として直交座標系に変換すると
    \begin{gather}
      \frac{(\rho\cos\phi)^2}{a^2} + \frac{(\rho\sin\phi)^2}{b^2} = 1 \\
      \frac{x^2}{a^2} + \frac{y^2}{b^2} = 1
    \end{gather}
    より、これは楕円を表す。
    また、$y$について解くと、
    \begin{gather}
      y = \pm\frac{b}{a} \sqrt{a^2 - x^2}
    \end{gather}
    概形は下図。
    % pic 楕円
    \item
    面積$S$は$x,y\geq 0$の領域$D$の面積の4倍として、
    この領域の境界は\dm{y = \frac{b}{a}\sqrt{a^2 - x^2}}だから
    \begin{align}
      S
      &= 4\int_{D} \,dxdy \\
      &= 4\int_{0}^{a}\frac{b}{a}\sqrt{a^2 - x^2} \,dx \\
      &= \pi ab
    \end{align}
  \end{enumerate}
\end{ans*}

\prob{
  次の微分方程式を解きなさい。
  \begin{enumerate}[label=(\alph*)]
    \item \dm{x\tan\Bigl(\frac{y}{x}\Bigr) - y + x\Bigl(\frac{dy}{dx}\Bigr) = 0}
    \item \dm{y\cos x \,dx + (2y+\sin x)dy = 0}
  \end{enumerate}
}
\begin{ans*}
  ${}$
  \begin{enumerate}[label=(\alph*)]
    \item 同次形より$y = ux$とおく。\dm{\frac{dy}{dx} = \frac{du}{dx}x + u}を用いて
    \begin{gather}
      \tan u + x\frac{du}{dx} = 0 \\
      \frac{du}{\tan x} = \frac{dx}{x} \\
      \log\left|\frac{y}{x}\right| = \log |x| + C \:\:(\text{$C$は任意定数})
    \end{gather}
    \item
    \begin{align}
      \int y\cos x \,dx
      &= y \sin x + f(y) \\
      \int (2y + \sin x) \,dy
      &= y^2 + y\sin x + g(x)
    \end{align}
    ただし、$f(y),\,g(x)$は任意に取れる関数である。

    よって、関数\dm{U(x,y) = y^2 + y\sin x}とおけば与えられた微分方程式は
    完全微分方程式で、
    \begin{gather}
      y\cos x \,dx + (2y+\sin x)dy = 0 \\
      \eqa dU = \ppar{U}{x}dx + \ppar{U}{y}dy = 0 \\
      \therefore y^2 + y\sin x = C \:\:(\text{$C$は任意定数})
    \end{gather}
  \end{enumerate}
\end{ans*}


\subsection{線形代数}

\prob{
  $x^2+y^2=1$を満たす点\dm{P=\vec{x \\ y}}
  の集合は円$\grG_0$である。
  点$\bP$を以下の行列$\bA$
  \begin{gather}
    \bA = \frac{1}{2}\pmat{3 & 1 \\ 1 & 3}
  \end{gather}
  を表現行列とする線形変換$L_{\bA}$で変換すると、
  その集合は閉曲線$\grG_1$となる。
  \begin{enumerate}[label=(\arabic*)]
    \item 行列$\bA$の固有値と正規化された固有ベクトルをすべて求めよ。
    \item $x\text{-}y$平面上に円$\grG_0$と閉曲線$\grG_1$を描け。
  \end{enumerate}
}
\begin{ans*}
  ${}$
  \begin{enumerate}[label=(\arabic*)]
    \item 固有方程式$\det(\grl\bE - \bA)=0$より
    \begin{align}
      \det(\grl\bE - \bA)
      &= \biggl(\frac{3}{2} - \grl\biggr)\biggl(\frac{3}{2} - \grl\biggr) - \frac{1}{4} \\
      &= (\grl - 2)(\grl - 1) = 0
    \end{align}
    \begin{gather}
      \therefore \grl = 2,\,1
    \end{gather}
    \begin{enumerate}[label=(\roman*)]
      \item $\grl = 2$のとき、固有ベクトル$\bu_1$は
      \begin{gather}
        \bmat{
          \disp -\frac{1}{2} & \disp\frac{1}{2} \\
          & \\
          \disp \frac{1}{2} & \disp -\frac{1}{2}
        }\bu_1 = \bzv \\
        \bu_1 = \frac{1}{\sqrt{2}}\Bmat{
          1 \\ 1
        }
      \end{gather}
      \item $\grl = 2$のとき、固有ベクトル$\bu_2$は
      \begin{gather}
        \bmat{
          \disp \frac{1}{2} & \disp \frac{1}{2} \\
          & \\
          \disp \frac{1}{2} & \disp \frac{1}{2}
        }\bu_1 = \bzv \\
        \bu_1 = \frac{1}{\sqrt{2}}\Bmat{
          -1 \\ 1
        }
      \end{gather}
    \end{enumerate}
    \item
    線形変換$L_{\bA}$によって点$\bP$が$\bQ = \{X,\,Y\}^{\top}$に移る、すなわち
    \begin{gather}
      \Bmat{X \\ Y} = \bA\Bmat{x \\ y} \\
    \end{gather}
    が成り立つとき、
    \begin{align}
      \Bmat{x \\ y}
      &= \bA^{-1}\Bmat{X \\ Y} \\
      &= \frac{1}{2}\bmat{
        \disp\frac{3}{2} & \disp -\frac{1}{2} \\
        &\\
        \disp -\frac{1}{2} & \disp \frac{3}{2}
      }\Bmat{X \\ Y} \\
      \Bmat{x & y}
      &= \Bmat{X & Y}(\bA^{-1})^{\top} \\
      &= \Bmat{X & Y}\frac{1}{2}\bmat{
        \disp\frac{3}{2} & \disp -\frac{1}{2} \\
        &\\
        \disp -\frac{1}{2} & \disp \frac{3}{2}
      }
    \end{align}
    また、$\grG_{0}$の方程式は
    \begin{gather}
      \Bmat{x & y} \bE \Bmat{x \\ y} = 1
    \end{gather}
    であるのでここに線形変換$L_{\bA}$を考えれば、
    \begin{gather}
      \Bmat{X & Y} (\bA^{-1})^{\top} \bA^{-1} \Bmat{X \\ Y} = 1 \\
      \therefore
      \Bmat{X & Y} \bmat{
        \disp \frac{5}{8} & \disp -\frac{3}{8} \\
        & \\
        \disp -\frac{3}{8} & \disp \frac{5}{8}
      }\Bmat{X \\ Y} = 1
    \end{gather}
    を満たす。よって線形変換によって移る点の集合は次の方程式で表される。
    \begin{gather}
      \Bmat{x & y} \bmat{
        \disp \frac{5}{8} & \disp -\frac{3}{8} \\
        & \\
        \disp -\frac{3}{8} & \disp \frac{5}{8}
      }\Bmat{x \\ y} = 1
    \end{gather}
    ここで、行列\dm{\bB := \bmat{
      \disp \frac{5}{8} & \disp -\frac{3}{8} \\
      % & \\
      \disp -\frac{3}{8} & \disp \frac{5}{8}
    }}
    は次のように対角化される。
    \begin{align}
      \det(\bB - \grl\bE)
      &= \biggl(\frac{5}{8} - \grl\biggr)\biggl(\frac{5}{8} - \grl\biggr) - \frac{9}{64} \\
      &= (\grl - 1)\biggl(\grl - \frac{1}{4}\biggr) = 0
    \end{align}
    \begin{equation}
      \therefore
      \grl = 1,\,\frac{1}{4}
    \end{equation}
    固有ベクトル$\bu_1,\,\bu_2$は
    \begin{align}
      \grl = 1 \text{\:のとき} & \bu_1 = \frac{1}{\sqrt{2}}\Bmat{-1 \\ 1} \\
      \grl = \frac{1}{4} \text{\:のとき} & \bu_2 = \frac{1}{\sqrt{2}}\Bmat{-1 \\ -1}
    \end{align}
    より$\grL:=[\bu_1,\,\bu_2]$とおくと$\grL\bB\grL^{-1}$は対角行列で
    \begin{gather}
      \grL\bB\grL^{-1} = \bmat{
        \disp 1 & 0 \\
        & \\
        0 & \disp\frac{1}{4}
      }
    \end{gather}
    である。また、\dm{
      \grL =
      \bmat{
        \disp\cos\frac{3}{4}\pi & \disp -\sin\frac{3}{4}\pi \\
        & \\
        \disp\sin\frac{3}{4}\pi & \disp\cos\frac{3}{4}\pi
      }
    }から、これは\dm{\frac{3}{4}\pi}方向の回転行列で
    $x\text{-}y$座標から\dm{\frac{3}{4}\pi}回転した$s\text{-}t$座標で
    \begin{gather}
      \Bmat{s & t} \bmat{
        \disp 1 & 0 \\
        & \\
        0 & \disp\frac{1}{4}
      } \Bmat{s \\ t} = 1 \\
      s^2 + \frac{t^2}{4} = 1
    \end{gather}
    より、楕円を表す。
  \end{enumerate}
  % pic 円と楕円
\end{ans*}

\prob{
  連立方程式
  \begin{align}
    \begin{aligned}
      x + 2y + 4z &= 3 \\
      x + 3y + 7z &= 0 \\
      x + y + z &= c
    \end{aligned}
  \end{align}
  に関する以下の問いに答えよ。
  ここに、$c$は実数である。
  \begin{enumerate}[label=(\arabic*)]
    \item 連立方程式が解を持つための必要十分条件を示せ。
    \item (1)の条件のもとで連立方程式の解を求めよ。
  \end{enumerate}
}
\begin{ans*}
  ${}$
  \begin{enumerate}[label=(\arabic*)]
    \item 係数と右辺を取り出した行列を$\bA,\,\bb$とする。
    \begin{gather}
      \bA = \bmat{
        1 & 2 & 4 \\
        1 & 3 & 7 \\
        1 & 1 & 1
      },\,
      \bb = \Bmat{
        3 \\ 0 \\ c
      }
    \end{gather}
    このとき、解を持つ必要十分条件は$\rank\bA = \rank[\bA|\bb]$が成り立つことである。
    \begin{align}
      [\bA|\bb]
      &\lra \left[\begin{array}{ccc|c}
        1 & 2 & 4 & 3 \\
        0 & 1 & 3 & -3 \\
        0 & -1 & -3 & c-3
      \end{array}\right] \\
      &\lra \left[\begin{array}{ccc|c}
        1 & 2 & 4 & 3 \\
        0 & 1 & 3 & -3 \\
        0 & 0 & 0 & c-6
      \end{array}\right]
    \end{align}
    ゆえに、求める条件は$c = 6$
    \item $[\bA|\bb]$について掃き出し法によって解を求める。
    \begin{gather}
      [\bA|\bb]
      \lra \left[\begin{array}{ccc|c}
        1 & 2 & 4 & 3 \\
        0 & 1 & 3 & -3 \\
        0 & 0 & 0 & 0
      \end{array}\right] \\
    % \end{gather}
    % \begin{gather}
      \begin{dcases}
        x + 2y + 4z = 3 \\
        y + 3z = -3
      \end{dcases}
    \end{gather}
    \begin{gather}
      \therefore
      \Bmat{x \\ y \\ z} =
      \Bmat{
        3-6t \\ -3 -3t \\ t
      }\quad(t\in\bbR)
    \end{gather}
  \end{enumerate}
\end{ans*}
