\newpage
\section{未定係数法}
$y''+ay'+by=r(x)$のような微分方程式を
非斉次常微分方程式という。
通常、非斉次の場合は次のようにする。

\begin{table}[H]
\centering
\begin{tabular}{ccccc}
\toprule
  右辺$r(x)\propto$ & 特解$y_{\rm{p}}$ \\
\midrule
  $e^{ax}$ & $k e^{ax}$ \\
  $x^n$ & $k + k_1 x + \cdots + k_n x^n$ \\
  $\sin{ax},\cos{ax}$ & $k_1 \cos{ax} + k_2 \sin{ax}$ \\
\bottomrule
\end{tabular}\end{table}

ただし一般解にこれらが含まれている場合はその限りではない。
その場合は一般解が含まれなくなるまで$x$倍するような工夫が必要である。
たとえば、右辺が$e^x$であるような斉次方程式では
基本解として$e^x$と$e^{2x}$が得られたときには
特解を$xe^x$とおき、
基本解として$e^x$と$xe^{x}$が得られたときには
特解を$x^2e^x$とおく。

この方法を未定係数法といい、定数係数についての方程式を解けば特殊解が得られる。

未定係数法は適切に解の形を仮定しなくてはならないため、
右辺によっては良い仮定が見つからないことがある。
そこで定数変化法を用いることがある。
