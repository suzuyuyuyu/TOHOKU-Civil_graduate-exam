\section{ベルヌーイ(Bernoulli)型\cdot リッカチ(Riccati)型}
一階線形非斉次常微分方程式において、
右辺が$Q(x)y^n$になっているものをベルヌーイ型、
$f(x)y^2 + g(x)y + h(x)$になっているものをリッカチ型という。
\begin{gather}
  \frac{dy}{dx} + P(x)y = Q(x)y^n \label{eq:appendix-ODE-Bernoulli} \\
  \frac{dy}{dx} + P(x)y = f(x)y^2 + g(x)y + h(x) \label{eq:appendix-ODE-Riccati}\\
  \frac{dy}{dx} + P(x)y = Q(x) (\text{積分因子法によって解く形})\\
\end{gather}

ベルヌーイ型は$u=y^{1-n}$とおくことで$u'=(1-n)y^{-n}y'$
であり、\refeq{eq:appendix-ODE-Bernoulli}の両辺に$(1-n)y^{-n}$
をかければ
\begin{gather}
  u' + (1-n)P(x)u = (1-n)Q(x)
\end{gather}
として一階線形微分方程式となる。


