\newpage
\section{2023春}

\subsection{微分積分}
\prob{%
  次の極限値を求めよ。
  ただし、\dm{a>0,b>0}とする。
  \begin{gather}
    \lim_{x\to 0}\left(\frac{a^x+b^x}{2}\right)^{1/x}
  \end{gather}
}

\begin{ans*}
  \dm{\frac{a^x + b^x}{2}>0}より
  与えられた極限に対数をとった次の極限
  \begin{gather}
    \lim_{x\to 0} \frac{\log \cfrac{a^x + b^x}{2}}{x}
  \end{gather}
  を考える。
  この分子
  \dm{f(x) = \log \frac{a^x + b^x}{2}}
  と分母
  \dm{g(x) = x}
  は微分可能でこれを微分した極限は
  \begin{align}
    \lim_{x\to 0} \frac{f'(x)}{g'(x)}
    &=\lim_{x\to 0} \frac{\cfrac{1}{2}\;(a^x\log a+b^x\log b)}{\cfrac{a^x + b^x}{2}} \\
    &= \frac{\log a + \log b}{2} = \log \sqrt{ab}
  \end{align}
  と得られる。
  ここに、
  $\disp\forall x\in\bbR,\,\disp g'(x) = 1 \neq 0 $と$ \lim_{x\to 0}f(x) = \lim_{x\to 0}g(x) = 0$
  であることから\lhopital の定理\footnote{appendix}より
  極限
  \dm{
    \lim_{x\to\infty}\frac{f(x)}{g(x)}
  }
  は存在して
  \begin{gather}
    \lim_{x\to 0} \frac{\log \cfrac{a^x + b^x}{2}}{x} = \log\sqrt{ab}
  \end{gather}
  ゆえに、求める極限は
  \begin{align}
    \lim_{x\to 0}\left(\frac{a^x+b^x}{2}\right)^{1/x}
     = \sqrt{ab}
  \end{align}
  である。
\end{ans*}

\prob{%
  \dm{x=r\cos\theta,\,y=r\sin\theta}のとき、以下の問いに答えよ。

  \begin{enumerate}[label=(\arabic*)]
    \item $dx$および$dy$を求めよ。
    \item $r,\,dr,\,\theta,\,d\theta$を用いて次式を表せ。
    \begin{gather}
      x\,dy - y\,dx
    \end{gather}
  \end{enumerate}
}
\begin{ans*}
  ${}$
  \begin{enumerate}[label=(\arabic*)]
    \item
    \begin{gather}
      dx = \cos\theta\,dr - r\sin\theta\,d\theta \\
      dy = \sin\theta\,dr + r\cos\theta\,d\theta
    \end{gather}
    \item
    \begin{align}
      \begin{split}
        x\,dy - y\,dx
        &= r\cos\theta(\sin\theta \,dr + r\cos\theta\,d\theta) \\
        &\hspace*{30pt} - r\sin\theta(\cos\theta\,dr - r\sin\theta\,d\theta)
      \end{split}\\
      &= r^2\cos^2\theta\,d\theta + r^2\sin\theta\,d\theta = r^2\,d\theta
    \end{align}
  \end{enumerate}
\end{ans*}


\prob{%
  次に示す$xyz$空間の領域$D$を考える。
  このとき、以下の問いに答えよ。
  \begin{gather}
    D = \Dset{(x,y,z)\relmiddle y\geq 0,\, z\geq 0,\, z^2\leq 4x,\, y^2\leq x-x^2}
  \end{gather}

  \begin{enumerate}[label=(\arabic*)]
    \item 領域$D$を図示せよ。
    \item 領域$D$の体積を求めよ。
  \end{enumerate}
}
\begin{ans*}
  ${}$
  \begin{enumerate}[label=(\arabic*)]
    \item 略 % pic
    \item \dm{
      C = \Dset{(x,y)\relmiddle \left(x-\frac{1}{2}\right)^2 + y^2 \leq \left(\frac{1}{2}\right)^2}
      }として、
    求める体積$V_{\rm{D}}$は
    \begin{align}
      V_{\rm{D}}
      &= \iint_C z\,dxdy \\
      &= \iint_C 2\sqrt{x}\,dxdy \\
      &= \int_{0}^{1}\int_{0}^{\sqrt{x-x^2}} 2\sqrt{x}\,dydx \\
      &= \int_{0}^{1}2\sqrt{x}\sqrt{x-x^2}\,dx \\
      &= \int_{0}^{1} 2x\sqrt{1-x}\,dx \\
      &= \int_{0}^{1} 2\sqrt{1-x}\,dx - \int_{0}^{1}2(1-x)\sqrt{1-x}\,dx \\
      &= \frac{8}{15}
    \end{align}
  \end{enumerate}

\end{ans*}

\prob{%
  次の微分方程式を解け。
  ここで、$e$は自然対数の底である。
  \begin{gather}
    y'' - 2y' + 2y - e^x - 2x = 0
  \end{gather}
}
\begin{ans*}
  斉次の微分方程式$y'' - 2y' + 2y = 0$の解は
  特性方程式$\grl^2 - 2\grl + 2 = 0\Longleftrightarrow \grl = 1\pm i$より
  \begin{gather}
    y = e^x (A\cos x + B\sin x)
  \end{gather}
  ここで、与えられた微分方程式について特解を\dm{y_{\rm{p}} = ke^x + ax + b}と仮定する\footnote{appendix}と
  \begin{gather}
    \begin{dcases*}
      y'_{\rm{p}} = ke^x + a \\
      y''_{\rm{p}} = ke^x
    \end{dcases*}
  \end{gather}
  であるので、
  \begin{gather}
    ke^x - 2(ke^x + a) + 2(ke^x + ax + b) = e^x + 2x \\
    \therefore k = a = b = 1
  \end{gather}
  ゆえに与えられた微分方程式の一般解は
  \begin{gather}
    y = e^x(A\cos x + B\sin x) + e^x + x + 1
  \end{gather}
\end{ans*}


\subsection{線形代数}
\prob{%
  以下の線形方程式の基本解を求めよ。
  \begin{align}
    \begin{aligned}
      x_1 + x_2 + x_3 + 2x_4 &= 0 \\
      x_1 + 2x_2 + 3x_4 &= 0 \\
      x_1 + 3x_2 - x_3 + 4x_4 &= 0
    \end{aligned}
  \end{align}
}
\begin{ans*}
  \begin{gather}
    \bA := \bmat{
      1 & 1 & 1 & 2 \\
      1 & 2 & 0 & 3 \\
      1 & 3 & -1 & 4
    },\quad \bx := \Bmat{
      x_1 \\ x_2 \\ x_3 \\ x_4
    }
  \end{gather}
  とおくと与えられた方程式は
  \begin{align}
    \bA \bx = \bm{0}
  \end{align}
  このとき、\bA の行基本変形によって
  \begin{gather}
    \bA \longrightarrow
    \bmat{
      1& 1& 1& 2 \\
      0& 1& -1& 1 \\
      0& 0& 0& 0 \\
    }
  \end{gather}
  とできるので、
  \begin{align}
    &\left\{
    \begin{aligned}
      x_1 + x_2 + x_3 + 2x_4 &= 0 \\
      x_2 - x_3 + x_4 &= 0
    \end{aligned}
    \right.
  \end{align}
  よって、基本解は$\dim\bx - \rank\bA = 2$から2つあって
  $x_1,x_2 = s,t \in \mathbb{R}$として解は
  \begin{align}
    \bx = s
    \Bmat{
      1 \\ 0 \\ -1/3 \\ -1/3
    }
    + t
    \Bmat{
      0 \\ 1 \\ 1/3 \\ -2/3
    }
  \end{align}
  と表せる。基本解は
  \begin{align}
    \Bmat{
      1 \\ 0 \\ -1/3 \\ -1/3
    },\,
    \Bmat{
      0 \\ 1 \\ 1/3 \\ -2/3
    }
  \end{align}
\end{ans*}


\prob{%
  行列$\bA$について以下の問いに答えよ。
  \begin{gather}
    \bA =
    \pmat{
      1 & 3 & 2 \\
      0 & 2 & 0 \\
      1 & 1 & 0
    }
  \end{gather}

  \begin{enumerate}[label=(\arabic*)]
    \item $\bA$の階数を求めよ。
    \item $\bA$の行列式を求めよ。
    \item $\bA$のすべての固有値を求めよ。
    \item $\bA$の各固有値に対する固有ベクトルを求めよ。
    \item 行列$\bA$は対角化可能か否かを調べよ。
  \end{enumerate}
}
\begin{ans*}
  ${}$
  \begin{enumerate}[label=(\arabic*)]
    \item 行基本変形によって
    \begin{gather}
      \bA \longrightarrow
      \bmat{
        1& 0& 0 \\
        0& 1& 0 \\
        0& 0& 1
      }
    \end{gather}
    とできるので$\rank \bA = 3$
    \item 余因子展開から
    \begin{align}
      \det \bA
      &=
      \vmat{
        2& 0 \\
        1& 0
      }
      - 0
      \vmat{
        3& 2 \\
        1& 0
      }
      +
      \vmat{
        3& 2 \\
        2& 0
      }
      = -4
    \end{align}
    \item 固有方程式$\det(\grl\bE - \bA) = 0$より
    \begin{align}
      \det
      \bmat{
        \grl-1& 3& 2 \\
        0& \grl-2& 0 \\
        1& 1& \grl
      }
      &= (\grl -1)
      \vmat{
        \grl-2 & 0 \\
        1& \grl
      }
      +
      \vmat{
        3& 2 \\
        \grl-2& 0
      } \\
      &= (\grl-1)(\grl-2)\grl - 2(\grl-2) \\
      &= (\grl-2)^2(\grl+1) = 0
    \end{align}
    \begin{gather}
      \therefore \grl = -1, 2
    \end{gather}
    \item 固有ベクトルを$\bu = \{x,\,y,\,z\}^\top$とする。
    \begin{enumerate}[label=(\roman*)]
      \item $\grl = -1$のとき
      \begin{gather}
        (\grl\bE - \bA)\bu = \bm{0} \\
        \bmat{
          -2& -3& -2 \\
          0& -3& 0 \\
          -1& -1& -1 \\
        }
        \Bmat{
          x \\ y \\ z
        }
        =\bm{0}
      \end{gather}
      よって、固有ベクトルは
      \begin{gather}
        \bu =
        \Bmat{
          1 \\ 0 \\ -1
        }
      \end{gather}
      \item $\grl = 2$のとき
      \begin{gather}
        (\grl\bE - \bA)\bu = \bm{0} \\
        \bmat{
          1& -3& -2 \\
          0& 0& 0 \\
          -1& -1& 2
        }
        \Bmat{
          x \\ y \\ z
        }
        =\bm{0}
      \end{gather}
      よって、固有ベクトルは
      \begin{gather}
        \bu =
        \Bmat{
          2 \\ 0 \\ 1
        }
      \end{gather}
    \end{enumerate}
    \item 固有方程式の解のうち重解$\grl = 2$において固有ベクトルが1つなので対角化可能でない。
  \end{enumerate}
\end{ans*}
