\newpage
\section{2022春}
\setcounter{yearcounter}{2022}


\subsection{構造工学}
図-1の一様断面で曲げ剛性$EI$のはり部材PRと軸剛性$EA$のトラス部材QSからなる骨組構造について、
以下の問いに答えなさい。ただし、部材QSの両端は摩擦のないヒンジとする。解答にあたっては、
次の関係式を用いて断面積Aを消去しなさい。
\begin{align*}%\label{eq:}
  EAl^2 = EI
\end{align*}
ここで、$E$はYoung率、$I$は断面二次モーメントである。
\begin{enumerate}
  \item 点Rに、鉛直下向きの荷重$P_0>0$と時計回りモーメントの荷重$M_0>0$が作用するとき、
        トラス部材QSの軸力を求めなさい。
  \item (1)のとき、はり部材PRの曲げモーメント図を描きなさい。ただし、P、Q、Rの各点での値も明記しなさい。
  \item (1)のとき、最大の引張応力が生じる$x$座標とその応力値を求めなさい。
        なお、はりの断面は高さ$h$、幅$b$の長方形として$I = bh^3/12$を用いなさい。
  \item $P_0 = 0$、$M_0 = 0$のとき、点Rのたわみを求めなさい。
  \item (3)で求めた最大の引張応力が作用する面が反時計回りに$60^\circ$回転した面に作用するせん断応力を求めなさい。
\end{enumerate}

\subsection{コンクリート工学}

表-1はコンクリートの示方配合表である。表中の(a)、(b)、(c)にあてはまる数値を答えよ。
水、セメント、細骨材、粗骨材の密度はそれぞれ
$d_w=1.00\:\text{g/cm}^3$、$d_c=3.10\:\text{g/cm}^3$、$d_s=2.50\:\text{g/cm}^3$、$d_g=2.50\:\text{g/cm}^3$、とする。

% 表

コンクリート用骨材の主要な特性を3つ示し、それぞれの特性がフレッシュコンクリート及び硬化コンクリートに及ぼす影響を説明せよ。

図-1に示す断面が一様である鉄筋コンクリートはりについて以下の問いに答えよ。
\begin{enumerate}[(1)]
  \item スパン中央の最大曲げモーメント$M_{max}$を答えよ。
  \item 曲げひび割れが初めて生じるときの荷重$P_{cr}$を答えよ。
        コンクリートの曲げ強度は$f_b$とする。ただし、軸方向鉄筋の影響は無視してよい。
\end{enumerate}

\subsection{地盤工学}
Tezaghiの一次元圧密方程式は次式で与えられる。
\begin{align*}%\label{eq:}
  \frac{\partial u_e}{\partial t} = c_v \frac{\partial^2u_e}{\partial z^2},\:
  c_v = \frac{k}{m_v \gamma_w}
\end{align*}

ここで、$u_e(z,t)$は地表面からの深さ$z$、時刻$t$における過剰間隙水圧、$c_v$は圧密係数、$k$は透水係数、
$m_v$は土骨格の体積圧縮係数、$\gamma_w$は水の単位体積重量である。
また、地表表面が排水条件、底面が非排水条件にある層厚$H$の地盤に対し、
等分布荷重$q$を瞬間載荷した場合を対象に式(1)を解くと、次式に示す級数解が得られる。
\begin{align*}%\label{eq:}
  u_e(z,t) = \sum_{n=1}^{x}\frac{2q}{\lambda_n}\sin(\lambda_n)\frac{z}{H}e^{-\lambda_n^2\frac{c_v}{H^2}t},\:
  \lambda_n = \frac{2n-1}{2}\pi
\end{align*}

式(1)および式(2)に基づいて以下の問いに答えよ。なお, Terzaghiの圧密理論における仮定は全て受け入れるものとする。
\begin{enumerate}[(1)]
  \item 時刻$t$における沈下量$\rho(t)$は過剰間隙水圧$u_e(z,t)$を用いて式(3)に示すように表される。
        \begin{align*}%\label{eq:}
          \rho(t) = m_v(qH - \int_{0}^{H}u_e(z,t)dz)
        \end{align*}
        載荷開始からの鉛直有効応力増分を$\Delta \sigma'(z,t)$、鉛直ひずみ$\varepsilon(z,t)$としたとき、
        $\varepsilon(z,t)=m_v\Delta \sigma'(z,t)$なる関係があることを用いて式(3)を導出せよ。
  \item 式(2)と式(3)より、$\rho(t)$を級数によって表せ。
  \item 設問(2)で求めた式より、最終沈下量$\rho_f$を求めよ。
  \item 設問(2)で求めた$\rho(t)$の級数表現の2次以上の項を無視することによって、沈下がある圧密度$U=(\rho(t)/\rho_f)$
        に達するまでの時間$t$を近似的に求める式を導け。
  \item 設問(4)で求めた近似式に基づいて、沈下が圧密度$U$に達するまでの時間に与える
        $H$、$m_v$、$k$および$q$の影響についてそれぞれ求めよ。
  \item 設問(4)で求めた近似式を用いて、時間係数$T_v(=c_vt/H^2)$と$U$の関係を表す近似式を求めよ。
        また、この近似式を用いて、$U$が0.9に達するときの$T_v$の値を求めよ。ただし、$ln(80/\pi^2)=2.09$を用いてよい。
\end{enumerate}