\newpage
\section{2021秋}
\setcounter{yearcounter}{2021}

\subsection{弾性体と構造の力学(1)}
\spprob{
  下に示すような、せん断弾性係数$\SI{50}{GPa}$の等方性$\cdot$線形弾性材料からなる
  中空断面のはりについて以下の問いに答えなさい。なお、この断面の外径と内径はそれぞれ
  $r_o=\SI{12}{cm}$、$r_i = \SI{10}{cm}$とする。また、このはりの中立軸は$x$軸に一致し、
  断面二次モーメントは次式で与えられるものとする。
  \begin{align*}%\label{eq:}
    I = \frac{\pi}{4}(r_o^4 - r_i^4)
  \end{align*}
  ここで、$\pi$は円周率である。なお解答は常分数で表せ。
  % PIC

  \begin{enumerate}[label=(\arabic*)]
    \item この断面に$z$軸周りの曲げモーメント$\pi \times10^4\si{N \cdot m}$が作用して、
          図中の点Pに$x$軸方向の最大引張応力が生じた。この応力を$\si{MPa}$の単位で答えなさい。
    \item 上記(1)の応力状態にある断面に、軸力($x$軸方向負荷)が作用して点Pの最大引張応力が
          \quad$\disp{\frac{2200\pi}{I}\times 10^{-6}}\si{MPa}$に変化した。ここで$I$は$\si{m^4}$の単位とする。
          このときに作用した軸力を$\si{kN}$の単位で求めなさい。
    \item このはりの$x$軸周りにねじりモーメント(トルク)\quad$T = 5\pi r^3(1-n^4) \times 10^6\si{N \cdot m}$
          が作用するとき、点Pにおける$yz$面内の最大(工学)せん断ひずみ$\gamma_{yz}$を求めなさい。
          なお、トルク$T$が$x$軸に作用するときの$yz$面内で最大せん断応力は次式で与えられる。
          \begin{align*}%\label{eq:}
            \tau_{yz} = \frac{2T}{\pi r_o^3(1-n^4)}
          \end{align*}
          ここで、$\disp{n = \frac{r_i}{r_o}}$である。
    \item この断面内で同時に上記の問い(2)、(3)の状態になるとき、図中の点Qにおけるにおける応力テンソルの成分を
          $3\times3$の行列表記で与えなさい。
    \item 上記(4)の応力状態について、最大、中間、最小主応力を求めなさい。
  \end{enumerate}
  % SI 単位系ミスアリ
}
\subsection{弾性体と構造の力学(2)}
\spprob{
  図-1~4に示す梁はすべて同じものである。図-1に示すように、単純梁のC点($x=l/2$)に単位の集中荷重を作用させたところ、
  $\disp{0\leq x \leq \frac{l}{2}}$の任意点$x$のたわみ$w(x)$が
  \begin{align*}%\label{eq:}
    w(x) = w_{max}\left\{3\frac{x}{l}-4\left(\frac{x}{l}\right)^3\right\}
  \end{align*}
  であった。ここに、$w_{max}$は最大たわみである。このことを用いて以下の問いに答えよ。
  ただし、梁の断面は軸方向に一様でたわみはは下向き正とする。

  \begin{enumerate}[label=\arabic*]
    \item 図-2に示す単純梁のたわみ$w_{2C}$を求めよ。
    \item 図-3に示す梁はC点をばね定数$k$のばねにより弾性支持されている。
          図-3のC点のたわみ$w_{3C}$が図-2のC点のたわみ$w_{2C}$の$\disp{\frac{1}{2}}$となるときのばね定数$k$を求めよ。
    \item 図-3に示すばねの内力(圧縮を正とする)を$f$とする。$\disp{\frac{f}{P}}$を求め、$\disp{\frac{f}{P}}$と$kw_{max}$との
          関係を図示せよ。
    \item 梁のC点をローラーヒンジ支点により支持した図-4に示す梁のC点の鉛直反力$V_C$(上向きを正とする)を求めよ。
  \end{enumerate}

  % PIC 
}
\subsection{地盤とコンクリート(1)}
\spprob{
  \begin{enumerate}[label=\arabic*.]
    \item 化学が関係する地盤工学に関わる課題を一つ挙げ、300字以内で説明せよ。
    \item 地盤工学が貢献できる持続可能な開発目標(SDGs)に関わる課題の内容を説明せよ。
    \item 図-1は、浸透流のある場合の飽和した半無限斜面内の応力を示している。斜面の安定性に関して、
          以下の問いに答えよ。なお土の湿潤単位体積重量$\gamma_t$、土の水中単位体積重量$\gamma_b$、
          水の単位体積重量$\gamma_w$とする。\\

          \begin{enumerate}[label=(\arabic*)]
            \item cd面に作用する直応力$\sigma$とせん断応力$\tau$を求めよ。
            \item cd面に作用する間隙水圧$u$と有効応力$\sigma'$を求めよ。
            \item 斜面のすべり安全率を示す式を示せ。ここに、土はせん断強度定数として、内部摩擦角$\varphi'$、
                  粘着力$c'$を持つものとする。
          \end{enumerate}
  \end{enumerate}

}
\subsection{地盤とコンクリート(2)}
\spprob{
  \begin{enumerate}[label=\arabic*.]
    \item せん断補強鉄筋の一つであるスターラップの主な役割について、次のキーワードを全て使用して説明せよ。\\

          キーワード$\colon$せん断力、骨材の嚙み合わせ、ダウエル作用\\

    \item 骨材がコンクリートのフレッシュ性状に及ぼす影響について次のキーワードを全て使用して説明せよ。\\
          
          キーワード$\colon$骨材の粒径、表面水率、細骨材率、単位水量、材料分離、スランプ\\

    \item 鉄筋コンクリート構造物の塩害に関する次の設問に答えよ。
        \begin{enumerate}[label=(\arabic*)]
          \item 鉄筋コンクリート構造物が健全な状態から体力が著しく低下するまでの劣化過程を説明せよ。
          \item 海洋環境下で生じる塩害と積雪寒冷地で生じる塩害の違いについて説明せよ。
        \end{enumerate}
  \end{enumerate}
}