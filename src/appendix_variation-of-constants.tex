\newpage
\section{定数変化法}
前節の方法は,非斉次項 $r(x)$ が比較的単純で,特解の関数形を事前に予測できる場合にしか使え
ない。一方,定数変化法を使うことで,任意の関数 $r(x)$ に対して非斉次方程式 
$y'' + ay' + by = r(x)$ の特解を求めるための公式を作ることができる。
ただし,できる限り前節の方法を使った方が簡単に特解を求められるので注意すること。

\begin{itembox}[l]{非斉次方程式の特解の公式}
  \begin{gather}
    y(x) 
    = -\left( \int_{x_1}^{x}\frac{y_2(\hat{x})r(\hat{x})}{W(\hat{x})}\:d\hat{x} \right) y_1(x)
    + \left(\int_{x_2}^{x}\frac{y_1(\hat{x})r(\hat{x})}{W(\hat{x})}\:d\hat{x}\right)y_2(x) \label{23}
  \end{gather}
  ここで,
  
  \begin{itemize}
    \item $y_1(x),y_2(x)\colon$非斉次方程式の互いに独立な解 
    \item $x_1,x_2\colon$任意の定数
    \item $W(x)$は以下で定義される$x$の関数(ロンスキアン)
    \begin{gather}
      W(x) := y_1(x)y'_2(x) - y'_1(x)y_2(x) \label{24}
    \end{gather}
  \end{itemize}
\end{itembox}


公式 \eqref{23} を得るためには,非斉次方程式 の特解
 $y(x)$ をあえて次の形に書き表すところ
から計算をスタートする。

\begin{gather}
  y(x) = C_1(x)y_1(x) + C_2(x)y_2(x) \label{25}
\end{gather}

ただし,$C_1(x), C_2(x)$ は任意関数で,$y'' + ay' + by=0$ が満たされるように決める。

この式は単に,未知関数 $r(x)$ を別の未知関数 $C_1(x), C_2(x)$ で書き換えているに過ぎない。
さらに,特解 $y(x)$ を書き換えるだけなら未知関数が 1 つだけあれば十分である。
そこで,2 つの未知関数 $C_1(x), C_2(x)$ を結びつける関係式 \eqref{27} を後ほど導入し,
未知関数の個数を減らすことにする。

特解を求めるためには,上記の $y(x)$ を非斉次方程式 $y'' + ay' + by=0$ に代入して,式が満たされるように
$C_1(x), C_2(x)$ を決めればよい。そのために,まず $y'(x)$ を計算すると

\begin{gather}
  y'(x) 
  = (C_1(x)y_1(x) + C_2(x)y_2(x))'  \notag\\
  = C'_1(x)y_1(x) + C'_2(x)y_2 + C_1(x)y'_1(x) + C_2(x)y'_2(x) \label{26}
\end{gather}

以下の計算では未知関数 $C_1(x), C_2(x)$ を求めていくことになるが,
その際に $C_1, C_2$ の微分項がなる
べく少ない方が計算が簡単になる。
そこで,$C_1(x), C_2(x)$ が次の関係式を満たすと仮定する:

\begin{gather}
  C'_1(x)y_1(x) + C'_2(x)y_2(x) = 0 \label{27}
\end{gather}


こう仮定すると,$C_1(x)$ と $C_2(x)$ のどちらかを決めればもう一方が決まるため,
式 \eqref{25} の右辺に
含まれる未知関数の個数が実質的に一つになる。

仮定 \eqref{27} を課すと,$y'(x)$ の表式は
\begin{gather}
  y'(x) = C_1(x)y'_1(x) + C_2(x)y'_2(x) \label{28}
\end{gather}

このとき,$y''(x)$ は
\begin{gather}
  y''(x) 
  = (C_1(x)y'_1(x) + C_2(x)y'_2(x))' \notag\\
  = C'_1(x)y'_1(x) + C'_2(x)y'_2(x) + C_1(x)y''_1(x) + C_2(x)y''_2(x) \label{29}
\end{gather}

式 \eqref{28}, \eqref{29} より,式 \eqref{25} の $y(x)$ を
非斉次方程式 $y'' + ay' + by=0$ に代入したものは

\begin{gather}
  r(x) 
  = y'' + ay' + by \notag\\
  = C'_1y'_1 + C'_2y'_2 + C_1y''_1 + C_2y''2 
  + a(C_1y'_1 + C_2y'_2)+ C_1y_1 + C_2y_2 \notag\\
  = C_1(y''_1 + ay'_1 + by_1)+ C_2(y''2 + ay'_2 + by_2)+ C'_1y'_1 + C'_2y'_2 \notag\\
  = C'_1y'_1 + C'_2y'_2 \notag\\
  \therefore C'_1y'_1 + C'_2y'_2 = r(x)
  \quad(\text{ただし\:} C'_1y_1 + C'_2y_2 = 0) \label{30}
\end{gather}

あとは,式 \eqref{30} を解いて $C_1(x), C_2(x)$ を決めればよい。
まず,\eqref{30} の 2 つの式を組み合わせて $C_1(x)$ だけの式を作る。
$C'_1y_1 + C'_2y_2 = 0$ より,
$C'_2 = -(y_1/y_2)C'_1$ と書き換えられるので

\begin{gather}
  r(x) 
  = C'_1y'_1 + C'_2y'_2 \notag\\
  = C'_1 y'_1 - \frac{y_1}{y_2} C'_1 y'_2 \notag\\
  = \frac{y'_1y_2 - y_1y'_2}{y_2}C'_1 \notag\\
  = -\frac{W(x)}{y_2(x)}C'_1(x) \label{31}\\
  \therefore C'_1(x) = -\frac{y_2(x)r(x)}{W(x)} \notag\\
  \Rightarrow C_1(x) 
  = -\int_{x_1}^{x} \frac{y_2(\hat{x})r(\hat{x})}{W(\hat{x})}\:d\hat{x} \label{32}
\end{gather}

ただし,$x_1$ は任意の定数で,$C'_1(x)$ の式を積分する際に生じる積分定数に相当する。
また,式を単純化するために,式 \eqref{24} で定義されるロンスキアン $W(x)$ を用いた。
同様に,$C_2(x)$ だけの式を作って積分すると

\begin{gather}
  C'_2(x) = \frac{y_1(x)r(x)}{W(x)}
  \Rightarrow C_2(x) =
  \int_{x_2}^{x}\frac{y_1(\hat{x})r(\hat{x})}{W(\hat{x})}\:d\hat{x} \label{33}
\end{gather}

先ほどと同様,$x_2$ は任意の定数である。
式 \eqref{32}, \eqref{33} を最初に仮定した特解の表式 \eqref{25} に代入して,公式 \eqref{23} を得る。
