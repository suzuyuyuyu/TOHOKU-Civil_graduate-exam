\newpage
\section{2021秋}
\setcounter{yearcounter}{2021}

\subsection{弾性体と構造の力学(1)}

等方均質な線形弾性体に底面応力状態下で一様な応力が作用している。
このときの応力成分$o-xy$座標系を参照して次のように与えられているとき、以下の問いに答えなさい。
\begin{align*}%\label{eq:}
  \begin{bmatrix}
    \sigma
  \end{bmatrix}
  =
  \begin{bmatrix}
    \sigma_x&\tau_{xy}\\
    \tau_{xy}&\sigma_y\\
  \end{bmatrix}
  =
  \begin{bmatrix}
    20 & 9\\
    9 & -4\\
  \end{bmatrix}
  \text{MPa}
\end{align*}

ここで、$\sigma_x$、$\sigma_y$はそれぞれ$x$と$y$方向の垂直応力、および$\tau_{xy}$はせん断応力である。
また、ベクトルの成分は$o-xy$座標系を参照しなさい。

\begin{enumerate}[(1)]
  \item 最大主応力と最小主応力を求め、それぞれの方向を表す単位ベクトルを求めなさい。
  \item 最大主応力と最小主応力が生じるそれぞれの面上の表面ベクトルを求めなさい。
  \item 最大主応力の方向と$x$軸とのなす角を$\theta$としたとき、$\tan \theta$を求めよ。解答に際しては倍角の公式
        \begin{align*}%\label{eq:}
          \tan 2\theta = \frac{2\tan \theta}{1 - \tan \theta^2}
        \end{align*}
        を用いてもよい。
  \item $xy$平面内の直線$4x + 3y +c = 0$の上での表面力ベクトルを求めなさい。なお$c$は任意の実数である。
  \item ヤング率$E$を200GPa、ポアソン比を0.2とするとき、最大および最小ひずみを求めよ。
        ただし、$x-y$座標系での応力とひずみの成分の間には以下の関係式があるものとする。
        \begin{align*}%\label{eq:}
          \varepsilon_x = \frac{\sigma_x}{E}-\nu\frac{\sigma_y}{E},\:
          \varepsilon_y = \frac{\sigma_y}{E}-\nu\frac{\sigma_x}{E},\:
          \gamma_{xy} = \frac{\tau_{xy}}{G}
        \end{align*}
        ここで、$\varepsilon_x$,$\varepsilon_y$はそれぞれ$x$と$y$方向の垂直ひずみ、
        $\gamma_{xy}$は$xy$面工学せん断ひずみである。
        また、Gはせん断弾性係数で$G=E/\{2(1+\nu)\}$で与えられる。
        なお、$z$軸方向垂直ひずみは考えなくてもよい。
\end{enumerate}

\subsection{弾性体と構造の力学(2)}
図-1~3に示す骨組構造の曲げ剛性は$EI$は一定であり、$P$は集中荷重である。
図-1の骨組構造の荷重載荷点C点のたわみ$w_{C1}$および支点Bの水平変位$u_{B1}$はそれぞれ
\begin{align*}%\label{eq:}
  w_{C1} = \frac{Pl^3}{48EI\cos\theta},\:
  u_{B1} = \frac{Pl^3\sin\theta}{24EI\cos^2\theta}
\end{align*}
である。ただし、たわみは下向きを正、水平変位は右向きを正とし、骨材を構成する部材の軸の伸び縮みは無視できるものとする。
図-1~3に示す骨組構造に関して以下の問いに答えよ。
\begin{enumerate}[1.]
  \item 図-2に示す骨組構造の支点Bの水平変位$u_{B2}$を求めよ。
  \item 図-2に示す骨組構造の点Cのたわみ$w_{C2}$を求めよ。
  \item 図-3に示す骨組構造の支点Bの水平反力$H_B$を求めよ。
  \item 図-3に示す骨組構造の荷重載荷点Cのたわみ$w_{C3}$を求めよ。
\end{enumerate}

\subsection{地盤とコンクリート(1)}

杭基礎が上部構造物を支持する機構について200字程度で説明せよ。式や図を用いてもよい。

土の水中単位重量$\gamma'$は土の飽和単位体積重量$\gamma_{sat}$、水の単位体積重量$\gamma_w$を用いて
$\gamma' = \gamma_{sat} - \gamma_w$と表せる。この式を基にして、$\gamma'$を間隙比$e$土粒子密度$\rho_s$、
水の密度$\rho_w$、重力加速度$g$の関数として表せ。

図-1のように、水深$10m$の海面下に砂地盤がある。海底面下$10m$のところから試料を採取して一連の三軸せん断試験したところ、
表-1の結果が得られた。以下の問いに答えよ。なお、表中の最大$\cdot$最小主応力は全応力表示である。

\begin{enumerate}[(1)]
  \item 試料を採取した地点の鉛直全応力、鉛直有効応力、水平有効応力を求めよ。
        なお水の単位体積重量は$9.8\:\text{kN/m}^3$、砂の飽和単位体積重量は$18.8\:\text{kN/m}^3$、
        静止土圧係数は$0.5$としてよい。
  \item 上記(1)および、せん断試験における破壊時の有効応力状態を示すモール円を描け。
  \item せん断強度をモール$\cdot$クーロンの破壊基準で表す場合、粘着力と内部摩擦角を求めよ。
  \item この土を対象に非排水繰返しせん断試験をした場合、どのような結果が予想されるかを考察せよ。
  \item この土を締め固めて再試験を行う時、試験結果がどう変化するか説明せよ。
\end{enumerate}

\subsection{地盤とコンクリート(2)}

暫定配合のコンクリートを試験練りしてスランプ試験と空気量試験を実施したところ、表-1の結果が得られた。
目標値を満たすように配合を修正するための方針について200字程度で説明せよ。

コンクリート用の混和剤に関する次の問いに答えよ。
\begin{enumerate}[(1)]
  \item フレッシュコンクリートの流動性を改善できる混和材を1つ挙げ、
        それが流動性を改善するメカニズムを100字程度で説明せよ。
  \item (1)で挙げた混和材がコンクリートの硬化後の瀬魚汁に及ぼす影響について、
        圧縮強度と物質の透過に対する抵抗性の観点から200字程度で説明せよ。
  \item (2)を踏まえて、(1)で挙げた混和材を使用するときの配合設計上の留意点を100字程度で説明せよ。
\end{enumerate}

引張鉄筋のみを有する単鉄筋長方形はりの曲げ破壊に関する次の問いに答えよ。
\begin{enumerate}[(1)]
  \item 「曲げ引張破壊」と「曲げ圧縮破壊」の破壊形態の違いについて100字程度で説明せよ。
  \item コンクリートの構造物の設計上、望ましくない破壊形態は曲げ引張破壊と曲げ圧縮破壊のどちらであるかを答えよ。
        また、その理由を200字程度で説明せよ。
\end{enumerate}